\chapter{Compactness of metric spaces}

\section{Compactness of metric spaces}

\begin{thm}{Exercise}
Assume a sequence of points $x_n$ in a metric space $\c{M}$ has no converging subsequence.
Show that $\c{M}$ is not compact.
(Hint: Note that for any $p\in \c{M}$ there is $r_p>0$ such that the ball $B(p,r_p)$ contains only finitely many elements of the sequence. Consider the covering of $\c{M}$ by balls $B(p,r_p)$.)
\end{thm}


\begin{thm}{Exercise}
Show that compact metric spaces has finite diameter;
that is, there is a pair of points which maximize the distace from eachother.
\end{thm}


Existace 
The following exercise describes another useful property of compact metric spaces;
the value $\eps>0$ is often useful 

Finally let me mention the so-called \emph{Lebesgue number}one useful construction


Let $\{V_\alpha\}$ be a cover of a metric space $\c{M}$.
A number $\eps > 0$ is called \emph{Lebesgue number} of the cover if for avery $x\in\c{M}$ there is $\alpha$ such that $ V_\alpha\supset \Ball(x,\eps)$.



\parit{Proof.}
Fix a cover $\{V_\alpha\}_{\alpha\in\c{I}}$ of a compact metric space $\c{M}$.

For any point $x\in \c{M}$, there is $\alpha$ such that $x\in V_\alpha$.
Moreover, since $V_\alpha$ is open there is $r_x>0$ such that 
$B(x,2\cdot r_x)\subset V_\alpha$.
Note that 
\[B(y,r_x)\subset V_\alpha\quad\text{for any}\quad y\in B(x,r_x).
\eqlbl{eq:B(y)inV}\] 

The collection of balls $B(x,r_x)$ as above for all $x$
is a cover of~$\c{M}$.
In particular, it admits a finite subcover
\[B(x_1,r_{x_1})\cup\dots\cup B(x_n,r_{x_n})\supset \c{M}.\eqlbl{eq:yinB(x)}\]

Set $\eps=\min\{r_{x_1},\dots,r_{x_n}\}$;
since $r_x>0$ for any $x$, we have that $\eps>0$.
From \ref{eq:B(y)inV}, $\eps$ is a Lebesgue number of the cover $\{V_\alpha\}_{\alpha\in\c{I}}$.
\qeds


\begin{thm}{Exercise}
Show that any topological space $\c{X}$ is homeomorphic to a subspace $\c{K}\setminus\{p\}$
where $\c{K}$ is compact and $p$ is a point in $\c{K}$.
\end{thm}

\section{Metric spaces}

The diameter of metric space is defined as the exact upper bound for the distances between pairs of its points;
the diameter of metric space $\c{X}$ is denoted as $\diam \c{X}$, it might take any value in $[0,\infty]$.

\begin{thm}{Exercise}
Show that any compact metric space have finite diameter.
In other words, show that if $\diam\c{X}=\infty$ then the metric space $\c{X}$ is not compact.

Moreover, any compact metric space $\c{X}$ there are two points $p$ and $q$ such that 
\[|p-q|=\diam \c{X}.\]
\end{thm}


Let $\c{X}$ be a metric space and $\{V_\alpha\}_{\alpha\in\c{I}}$ be its open cover.
A positive number $\delta$ is called Lebesgue number of the covering $\{V_\alpha\}$
if for any $x\in \c{X}$ there is $\alpha\in\c{I}$ such that $B(x,\delta)\subset V_\alpha$.

\begin{thm}{Theorem}
Any open cover of compact metric space has a Lebesgue number.
\end{thm}

\parit{Proof.}
Fix a compact metric space $\c{X}$ and its open cover $\{V_\alpha\}_{\alpha\in\c{I}}$.

Given $x\in \c{X}$, denote by $f(x)$ the exact upper bound for the values $r$ such that $B(x,r)\subset V_\alpha$ for some $\alpha\in \c{I}$.

Note that if $f(x)=\infty$ for some $x$ then one of $V_\alpha$ coincides with the whole space $\c{X}$.
In this case, any positive number is a Lebesgue number of $\{V_\alpha\}$.
Further we assume that $f(x)$ takes finite value for any $x\in\c{X}$.

Note that 
\[|f(x)-f(y)|\le |x-y|_{\c{X}};\]
in particular, the function $f\:\c{X}\to\RR$ is continuous.

Indeed, by triangle inequality, 
\[B(x,r)\supset B(y,r-|x-y|_{\c{X}}).\eqlbl{eq:lipschitz}\]
It follows that if $B(x,r)\subset V_\alpha$ for some $\alpha$,
then $B(y,r-|x-y|_{\c{X}})\subset V_\alpha$.
Therefore
\[f(x)-|x-y|_{\c{X}}<f(y).\]
Similarly, we get 
\[f(y)-|x-y|_{\c{X}}<f(x).\]
Hence \ref{eq:lipschitz} follows.

Since $f$ is continuous and $\c{X}$ is compact,
the function $f\:\c{X}\to \RR$ admits its minimum at some point $x_0$.

Note that $f(x_0)>0$.
Indeed, since $\{V_\alpha\}$ is an open cover of $\c{X}$,
there is $V_\alpha$ such that $V_\alpha\ni x_0$.
Since $V_\alpha$ is open, we have $V_\alpha\supset B(x_0,r)$ for some $r>0$.

Finally, note that by construction $\delta=\tfrac12\cdot f(x_0)$ is a Lebsgue number of $\{V_\alpha\}$. 
\qeds

\begin{thm}{Exercise}
Find an open cover of a compact metric space such that and $r<1$ is a Lebesgue number, but 1 is not its Lebesgue number.
\end{thm}
