\chapter{Fundamental group}

\section{Definition}

Let $\c{X}$ be a topological space.
A path $f\:[0,1]\to\c{X}$ is called a \emph{loop} with base at $p\in\c{X}$ if $f(0)=f(1)=p$;


Note that if $f$ and $g$ are loops based at $p$, then their products $f*g$, $g*f$ are defined and they are loops based at $p$ as well; see Section~\ref{sec:Operations on paths}.
Moreover, the time-reversed paths $\bar f$, $\bar g$ are also loops based at $p$.


Recall that $[f]$ denotes the homotopy class of $f$; for paths and loops homotopy always means homotopy relative to the ends. 
The multiplication of homotopy classes of loops based at $p$ is defined by 
\[[f]\cdot [g]=[f*g];\]
that is, \textit{the product of homotopy classes of loops $f$ and $g$ is the homotopy class of the product $f*g$.}

Observe that the product is well defined;
that is, if $[f_0]= [f_1]$ and $[g_0]=[g_1]$, then  $[f_0*g_0]=[f_1*g_1]$.
In other words, if $f_0\sim f_1$ and $g_0\sim g_1$ then $f_0*g_0\sim f_1*g_1$.
The latter is stated in Claim \ref{clm:product}.

Denote by $\pi_1(\c{X},p)$ the set of all homotopy classes of loops at $p$.

\begin{thm}{Theorem}
 $\pi_1(\c{X},p)$ with the introduced multiplication is a group.
\end{thm}

The group $\pi_1(\c{X},p)$ is called the \emph{fundamental group} of $\c{X}$ with base point $p$.

\parit{Proof.}
Recall that $\eps_p$ denotes the constant loop at $p$ in $\c{X}$;
that is, $\eps_p(t)=p$ for any $t$.
We will show that the homotopy class $[\eps_p]$ is the neutral element of $\pi_1(\c{X},p)$
and $[\bar f\,]=[f\,]^{-1}$, where $\bar f$ denoted the time-reversed $f$.

Note that conditions in the definition of group follow from the next three conditions for any loops $f$, $g$, and $h$ based at $p$ in $\c{X}$.
\begin{enumerate}[(i)]
\item $f*\eps_p\sim \eps_p*f\sim f$;
\item $f*\bar f\sim\bar f* f\sim \eps_p$;
\item $(f*g)*h\sim f*(g*h)$.
\end{enumerate}
These statements are provided by \ref{clm:neutral}, \ref{clm:f-bar-f}, and \ref{clm:assoc}.
\qeds

\section{Induced homomorphism}

Let $\phi\:\c{X}\to \c{Y}$ be a continuous map;
suppose $\phi(p)=q$.
If $f$ is a loop based at $p$ in $\c{X}$, then $\phi\circ f$ is a loop based at $q$ in $\c{Y}$.

The following claim implies that the map $f\mapsto\phi\circ f$ induces a homomorphism
\[\phi_*\:[f]\mapsto [\phi\circ f].\]

\begin{thm}{Claim}\label{clm:ind-hom}
Let $\phi\:\c{X}\to \c{Y}$ be a continuous map and $\phi(p)=q$.
Suppose that $f_0$ and $f_1$ are loops bases at $p$ in $\c{X}$.
Then

\begin{subthm}{clm:ind-hom-prod}
$\phi\circ(f_0*f_1)=(\phi\circ f_0)*(\phi\circ f_1)$,
\end{subthm}

\begin{subthm}{clm:ind-hom-homotopy}
if $f_0\sim f_1$, then $\phi\circ f_0\sim\phi\circ f_1$.
\end{subthm}

\end{thm}

\parit{Proof; \ref{SHORT.clm:ind-hom-prod}.}
Applying the definition of the product of paths and composition of maps to $\phi\circ(f_0*f_1)$ and $(\phi\circ f_0)*(\phi\circ f_1)$ we get exactly the same expression:
\begin{align*}
\begin{cases}
\phi\circ f_0(t)&\text{if}\ t\le \tfrac12,
\\
\phi\circ f_1(t)&\text{if}\ t\ge \tfrac12.
\end{cases}
\end{align*}
Hence \ref{SHORT.clm:ind-hom-prod} follows.

\parit{\ref{SHORT.clm:ind-hom-homotopy}.}
Observe that if $f_t$ is a homotopy from $f_0$ to $f_1$, then $\phi\circ f_t$ is a homotopy from $\phi\circ f_0$ to $\phi\circ f_1$.
Hence \ref{SHORT.clm:ind-hom-homotopy} follows.
\qeds



\begin{thm}{Exercise}
Consider continuous maps 
$\c{X}\stackrel{\phi}{\to}\c{Y}\stackrel{\psi}{\to}\c{Z}$ between topological spaces.
Show that $\psi_*\circ\phi_*=(\psi\circ\phi)_*$.
\end{thm}

\section{Dependence on base point}

\begin{thm}{Theorem}
Let $p$ and $q$ be two points in a topological space $\c{X}$.
Suppose there is a path $h$ from $p$ to $q$, then 
the fundamental groups $\pi_1(\c{X},p)$ and $\pi_1(\c{X},q)$ are isomorphic.
\end{thm}

According to the theorem, the fundamental group (more precisely its \textit{isomorphism class}) of path-connected space does not depend on its base point.
Therefore, for a path-connected space $\c{X}$ we do not need to specify the base point of its fundamental group; so we could write $\pi_1(\c{X})$ instead of $\pi_1(\c{X},p)$.

\parit{Proof.}
Suppose $f$ is a loop based at $p$.
Note that $\bar h*(f*h)$ is a loop at $q$.
Moreover, the map $f\mapsto \bar h*(f*h)$ induces a homomorphism $u_h\:\pi_1(M,p)\to\pi_1(M,q)$.

Indeed, suppose $f_t$ is a homotopy of loops at $p$.
Then $\bar h*(f_t*h)$ is a homotopy of loops at $q$.
It follows that the map 
\[u_h\:[f]\mapsto [\bar h*(f*h)]\]
is defined; that is, the right-hand side does not depend on the choice of loop $f$ in the homotopy class $[f]$. 

Further, if $f$ and $g$ are loops based at $p$, then \ref{clm:neutral}, \ref{clm:f-bar-f}, and  \ref{clm:assoc} imply that
\begin{align*}
(\bar h*(f*h))*(\bar h*(g*h))&\sim \bar h*(((f*(h*\bar h))*g)*h)\sim
\\
&\sim \bar h*(((f*\eps_p)*g)*h)\sim 
\\
&\sim \bar h*((f*g)*h).
\end{align*}
Whence the map $u_h\:\pi_1(M,p)\to \pi_1(M,q)$ is a homomorphism;
that is, 
\[u_h([f]\cdot [g])=u_h[f]\cdot u_h[g]\quad\text{for any}\quad [f],[g]\in \pi_1(\c{X},p).\]

The same argument shows that $u_{\bar h}\: \pi_1(M,q)\to \pi_1(M,p)$ defined by 
\[u_{\bar h}\:[k]\mapsto [h*(k*\bar h)]\]
is a homomorphism.
Note that 
\begin{align*}
h*((\bar h*(f*h))*\bar h)&\sim (h*\bar h)*(f*(h*\bar h))\sim
\\
&\sim \eps_p*(f*\eps_p)\sim
\\
&\sim f
\end{align*}
for any loop $f$ based at $p$.
Therefore, $u_{\bar h}$ is inverse of $u_h$.
The same way we show that $u_{h}$ is the inverse of $u_{\bar h}$.
It follows that $u_h$ is an isomorphism.
\qeds

\begin{thm}{Exercise}\label{ex:uphi}
Let $\phi_t\:\c{X}\to\c{Y}$ be a homotopy.
Suppose that $q_0=\phi_0(p)$ and $q_1=\phi_1(p)$; 
consider the path from $q_0$ to $q_1$ defined by $h(t)=\phi_t(p)$.
Show that 
\[u_h\circ \phi_{0*}=\phi_{1*}.\]
\end{thm}

\begin{thm}{Exercise}
Suppose that path-connected topological spaces $\c{X}$ and $\c{Y}$ have the same homotopy type.
Use \ref{ex:uphi} to show that their fundamental groups are isomorphic.
\end{thm}



\begin{thm}{Exercise}
Let $\c{X}$ and $\c{Y}$ be two path-connected topological spaces.
Choose points $p\in \c{X}$ and $q\in \c{Y}$.
Consider the projections $\phi\:\c{X}\times\c{Y}\to \c{X}$ and $\psi\:\c{X}\times\c{Y}\to \c{Y}$
and their induced homomorphisms
$\phi_*\:\pi_1(\c{X}\times\c{Y},(p,q))\to \pi_1(\c{X},p)$
and $\psi_*\:\pi_1(\c{X}\times\c{Y},(p,q))\to \pi_1(\c{Y},q)$.
Define $\Phi\:\pi_1(\c{X}\times\c{Y},(p,q))\to \pi_1(\c{X},p)\times \pi_1(\c{Y},q)$ by
\[\Phi\:\alpha\mapsto (\phi_*(\alpha),\psi_*(\alpha))\]
for any $\alpha\in \pi_1(\c{X}\times\c{Y},(p,q))$.
Note that the map $\Phi$ is a homomorphism.

\begin{subthm}{}
Show that $\Phi$ is a monomorphism;
that is, if $\Phi(\alpha)=\Phi(\beta)$ for some $\alpha,\beta\in \pi_1(\c{X}\times\c{Y},(p,q))$, then 
$\alpha=\beta$.
\end{subthm}

\begin{subthm}{}
Show that $\Phi$ is an epimorphism;
that is, for any $\gamma\in \pi_1(\c{X},p)\z\times\pi_1(\c{Y},q)$ there is $\alpha\in\pi_1(\c{X}\times\c{Y},(p,q))$ such that $\Phi(\alpha)=\gamma$.
\end{subthm}

Conclude that $\pi_1(\c{X}\times\c{Y})$ is isomorphic to $\pi_1(\c{X})\times\pi_1(\c{Y})$.
\end{thm}
