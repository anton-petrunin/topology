\chapter{Construction}

\section{Homotopy of paths}


\begin{wrapfigure}{r}{33 mm}
\vskip-4mm
\centering
\includegraphics{mppics/pic-80}
\end{wrapfigure}

Let $p$ and $q$ be two points in a topological sapce $\c{X}$ and
 $f_\tau\:[0,1]\to \c{X}$ be a one-parameter family of paths from $p$ to $q$; here $\tau\in [0,1]$.
If the map $[0,1]\times [0,1]\to \c{X}$, defined as $(\tau,t)\mapsto f_\tau(t)$ is continuous, then $f_\tau$ is called a \index{homotopy of paths}\emph{homotopy of paths} in $\c{X}$.

Intuitively, homotopy of paths is a path in the space of paths with fixed ends.
To make this statement precise, one has to introduce an appropriate topology on the space of all paths with fixed ends; the so-called \index{compact-open topology}\emph{compact-open topology} provides a right choice, but we are not going to touch this subject. 

Two paths $g,h\:[0,1]\to \c{X}$ are called {}\emph{homotopic} (briefly $g\sim h$)
if there is a homotopy $f_\tau\:[0,1]\to \c{X}$ such that $g=f_0$ and $h=f_1$.

Following the argument in \ref{sec:Operations on paths}, one can show that $\sim$ is an equivalence relation.
Therefore, we can talk about the equivalence class of a path $f$ that will be called its \index{homotopy class of paths}\emph{homotopy class};
it will be denoted by~$[f]$.

\begin{thm}{Exercise}
Suppose that $f$ and $g$ are two paths in $\RR$ with common ends;
that is, $f(0)=g(0)$ and $f(1)=g(1)$.
Show that $f\sim g$.
\end{thm}




\section{Technical claims}\label{sec:homotopy-claim}

\begin{thm}{Claim}\label{clm:product}
Suppose $f_0$ is a path from $p$ to $q$,
and 
$g_0$ is a path from $q$ to $r$.
Suppose $f_0\sim f_1$ and $g_0\sim g_1$,
then
\[f_0*g_0\sim f_1*g_1.\]
\end{thm}

\parit{Proof.}
Choose homotopies $f_\tau$ from $f_0$ to $f_1$, and $g_\tau$ from $g_0$ to $g_1$.
Observe that \ref{ex:closed-continuous} implies that
$f_\tau*g_\tau$ is a homotopy from $f_0*g_0$ to $f_1*g_1$.
\qeds

Each of the following claims proved by explicit construction of the needed homotopy.
Each time the homotopy constructed as a composition $h\circ s_\tau(t)$, where $h$ is a fixed path and  $s_\tau$ is a one-parameter family of functions $[0,1]\to [0,1]$.
The graphs of $s_\tau$ provide more intuitive descriptions of the families;
the formulas presented just to make it formally correct.

Recall that $\eps_p$ is the constant path with image $p$;
that is, $\eps_p(t)=p$ for any $t$.

\begin{thm}{Claim}\label{clm:neutral}
Suppose $f$ is a path from $p$ to $q$, then
\[\eps_p*f\sim f*\eps_q\sim  f.\]
\end{thm}

\begin{wrapfigure}{r}{30mm}
\centering
\vskip-5mm
\includegraphics{mppics/pic-30}
\end{wrapfigure}

\parit{Proof.}
Consider the function
\[s_\tau(t)=
\begin{cases}
2\cdot \tau \cdot t&\text{if}\ t\le \tfrac12,
\\
2\cdot\tau-1+2\cdot (1-\tau) \cdot t&\text{if}\ t\ge \tfrac12.
\end{cases}
\]
Observe that $(\tau,t)\mapsto s_\tau(t)$ and therefore $(\tau,t)\mapsto f(s_\tau(t))$ are continuous maps.
Therefore $h_\tau(t)=f(s_\tau(t))$ is a homotopy.

Further,
\begin{align*}
f(s_0(t))&=\eps_p*f(t),
\\
f(s_{\frac12}(t))&=f(t),
\\
f(s_1(t))&=f(t)*\eps_q
\end{align*}
for any $t$.
Whence the claim follows.
\qeds

\begin{thm}{Exercise}\label{ex:neutral}
Suppose that $f$ is a path in a Hausdorff space.
Assume $f(0)=p$, $f(1)=q$, and $\eps_p*f= f$.
Show that $f=\eps_p$; in particular, $p=q$.
\end{thm}

\begin{thm}{Claim}\label{clm:f-bar-f}
Suppose $f$ is a path from $p$ to $q$, then
\[f*\bar f\sim \eps_p
\quad\text{and}\quad
\bar f*f\sim \eps_q.\]
\end{thm}

\begin{wrapfigure}[4]{r}{30mm}
\centering
\vskip-10mm
\includegraphics{mppics/pic-31}
\end{wrapfigure}

\parit{Proof.}
Consider the function
\[s_\tau(t)=
\begin{cases}
2\cdot \tau \cdot t&\text{if}\ t\le \tfrac12,
\\
1-2\cdot \tau \cdot t&\text{if}\ t\ge \tfrac12.
\end{cases}
\]
Observe that $(\tau,t)\mapsto s_\tau(t)$ and therefore $(\tau,t)\mapsto f(s_\tau(t))$ are continuous maps.

Note that $f(s_1(t))=f*\bar f(t)$ for any $t$.
Therefore $h_\tau(t)=f(s_\tau(t))$ is a homotopy from $\eps_p$ to $f*\bar f$.

It proves the first statement.
The second statement follows from the first one since $\bar{\bar f}=f$.
\qeds

The product of paths is not associative;
that is, in general,
\[(f*g)*h\ne f*(g*h)\]
for paths $f,g,h$ such that both products are defined.
In other words, we have to specify the order of product.
The following claim says that \textit{product of paths is not associative up to homotopy}.

\begin{thm}{Claim}\label{clm:assoc}
Suppose $f$, 
$g$,
and
$h$ are paths such that $f(1)=g(0)$ and $g(1)=h(0)$.
Then
\[(f*g)*h\sim f*(g*h).\]
\end{thm}

\begin{wrapfigure}[3]{r}{30mm}
\centering
\vskip-4mm
\includegraphics{mppics/pic-32}
\end{wrapfigure}

\parit{Proof.}
Consider the function $s_\tau$ defined by 
\[s_\tau(t)=
\begin{cases}
\tfrac{1+\tau}2\cdot t&\text{if}\ t\le \tfrac12,
\\
\tfrac{\tau-1}4+t&\text{if}\ \tfrac 34\ge t\ge \tfrac12,
\\
\tau-1+ (2-\tau)\cdot t&\text{if}\  t\ge \tfrac34.
\end{cases}
\]
Observe that $(\tau,t)\mapsto s_\tau(t)$ and therefore $(\tau,t)\mapsto f*(g*h)(s_\tau(t))$ are continuous maps.

Note that $s_1(t)=t$,
and therefore \[f*(g*h)(t)\z=f*(g*h)(s_1(t))\] for any $t$.
Further, 
\[(f*g)*h(t)\z=f*(g*h)(s_0(t))\] for any $t$.
It remains to observe that $f*(g*h)(s_\tau(t))$ is the needed homotopy.
\qeds

\begin{thm}{Exercise}
Let $f$ and $g$ be paths from $p$ to $q$.
Show that $f\sim g$ if and only if $f*\bar g\sim \eps_p$.
\end{thm}


\begin{thm}{Advanced exercise}\label{ex:assoc}
Let $f$, $g$, and $h$ be paths in a Hausdorff space.
Suppose that $(f*g)*h= f*(g*h)$
and both sides of the equation are defined.
Show that $f=g=h=\eps_p$ for some point $p$.
\end{thm}


\section{Fundamental group}

Let $\c{X}$ be a topological space.
A path $f\:[0,1]\to\c{X}$ is called a \index{loop}\emph{loop} with base at $p\in\c{X}$ if $f(0)=f(1)=p$;


Note that if $f$ and $g$ are loops based at $p$, then their products $f*g$, $g*f$ are defined and they are loops based at $p$ as well; see Section~\ref{sec:Operations on paths}.
Moreover, the time-reversed paths $\bar f$, $\bar g$ are also loops based at $p$.


Recall that $[f]$ denotes the homotopy class of $f$; recall that homotopy of paths and loops does not move their ends. 
The multiplication of homotopy classes of loops based at $p$ is defined by 
\[[f]\cdot [g]=[f*g];\]
that is, \textit{the product of homotopy classes of loops $f$ and $g$ is the homotopy class of the product $f*g$.}

Observe that the product is well defined;
that is, if $[f_0]= [f_1]$ and $[g_0]=[g_1]$, then  $[f_0*g_0]=[f_1*g_1]$.
In other words, if $f_0\sim f_1$ and $g_0\sim g_1$ then $f_0*g_0\sim f_1*g_1$.
The latter is stated in Claim \ref{clm:product}.

Denote by $\pi_1(\c{X},p)$ the set of all homotopy classes of loops at $p$.

\begin{thm}{Theorem}
$\pi_1(\c{X},p)$ with the introduced multiplication is a group.
\end{thm}

The group $\pi_1(\c{X},p)$ is called the \index{fundamental grou}\emph{fundamental group} of $\c{X}$ with base point $p$.

\parit{Proof.}
Recall that $\eps_p$ denotes the constant loop at $p$ in $\c{X}$;
that is, $\eps_p(t)=p$ for any $t$.
We will show that the homotopy class $[\eps_p]$ is the neutral element of $\pi_1(\c{X},p)$
and $[\bar f\,]=[f\,]^{-1}$, where $\bar f$ denoted the time-reversed $f$.

Note that conditions in the definition of group follow from the next three conditions for any loops $f$, $g$, and $h$ based at $p$ in $\c{X}$.
\begin{enumerate}[(i)]
\item $f*\eps_p\sim \eps_p*f\sim f$;
\item $f*\bar f\sim\bar f* f\sim \eps_p$;
\item $(f*g)*h\sim f*(g*h)$.
\end{enumerate}
These statements are provided by \ref{clm:neutral}, \ref{clm:f-bar-f}, and \ref{clm:assoc}.
\qeds

\begin{thm}{Exercise}
Suppose that $V$ and $W$ are open subsets of topological space $\c{X}$ such that 
$\c{X}=V\cup W$, and the set $V\cap W$ is path-connected.
Let $p\in V\cap W$.
Show that any loop in $\c{X}$ based at $p$  is homotopic to a product of loops in $V$ or $W$ with the same base.
\end{thm}

\section{Simply-connected spaces}

Recall that a group is called \index{trivial group }\emph{trivial} if it contains only one element which is necessary the neutral element.

A path connected topological space with trivial fundamental group is called \index{simply-connected space}\emph{simply-connected}.

If the fundamental group $\pi_1(\c{X},p)$ is trivial, it is common to write $\pi_1(\c{X},p)=0$ despite that this equality does not have much sense --- in general the group $\pi_1(\c{X},p)$ is not commutative and so it would be more reasonable to write $\pi_1(\c{X},p)\z=\{1\}$, meaning that $1$ is the only element of $\pi_1(\c{X},p)$.



