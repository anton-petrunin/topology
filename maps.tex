\chapter{Maps}



\section{Homeomorphisms}

\begin{thm}{Definition}\label{def:homeo}
A bijection $f\:\c{X}\to\c{Y}$ between topological spaces 
is called \index{homeomorphism}\emph{homeomorphism} if $f$ and its inverse $f^{-1}\:\c{Y}\to\c{X}$
are continuous.%
\footnote{The term \textit{homomorphism} from abstract algebra looks similar and it has similar meaning but should not to be confused with \textit{homeomorphism}.}

Topological spaces $\c{X}$ and $\c{Y}$ are called \index{homeomorphic spaces}\emph{homeomorphic} (briefly, $\c{X}\simeq\c{Y}$) if there is a homeomorphism $f\:\c{X}\to\c{Y}$.
\end{thm}


\begin{thm}{Exercise}\label{ex:bijection-ne-homeo}
Show that any homomorphism is a continuous bijection.

Give an example of continuous bijection between topological spaces that is not a homeomorphism.
\end{thm}

\begin{thm}{Exercise}\label{ex:exp}
Show that $x\mapsto e^x$ is a homeomorphism $\RR\to (0,\infty)$.
\end{thm}

\begin{thm}{Exercise}\label{ex:arctan}
Construct a homeomorphism $f\:\RR\to (0,1)$.
\end{thm}

\begin{thm}{Exercise}\label{ex:homeo=eq}
Show that $\simeq$ is an equivalence relation;
that is, for any topological spaces $\c{X}$, $\c{Y}$, and $\c{Z}$ we have the following:
\begin{subthm}{}
$\c{X}\simeq\c{X}$;
\end{subthm}

\begin{subthm}{}
if $\c{X}\simeq\c{Y}$, then $\c{Y}\simeq\c{X}$;
\end{subthm}

\begin{subthm}{}
if $\c{X}\simeq\c{Y}$ and $\c{Y}\simeq\c{Z}$, then $\c{X}\simeq\c{Z}$.
\end{subthm}

\end{thm}

\begin{thm}{Advanced exercise}\label{ex:inversion}
Prove that the complement of a circle in the Euclidean space is homeomorphic to the Euclidean space without line $\ell$ and a point $p\not\in\ell$.
\end{thm}

\begin{wrapfigure}{r}{34 mm}
\vskip-4mm
\centering
\includegraphics{mppics/pic-70}
\end{wrapfigure}

Recall that a figure $F$ is called \index{star-shaped set}\emph{star-shaped} if there exists a point $p\in F$ such that for all $x\in F$ the line segment $px$ lies in~$F$.

\begin{thm}{Advanced exercise}\label{ex:star-shaped}
Any nonempty open star-shaped set in the plane is homeomorphic to the open disc.
\end{thm}

\begin{thm}{Advanced exercise}\label{ex:cont-dense}
Show that the complements of two countable dense subsets of the plane are homeomorphic.
\end{thm}

%???Restriction+embedding

\section{Closed and open maps}
\label{sec:Closed and open maps}

\begin{thm}{Definition}\label{def:open-closed-maps}
A map between topological spaces 
$f\:\c{X}\to\c{Y}$ is called \index{open map}\emph{open} if, for any open set $V\subset\c{X}$, the image $f(V)$ is open in~$\c{Y}$.

A map between topological spaces 
$f\:\c{X}\to\c{Y}$ is called \index{closed map}\emph{closed} if, for any closed set $Q\subset\c{X}$, the image $f(Q)$ is closed in $\c{Y}$.
\end{thm}

Note that homeomorphism can be defined as a \textit{continuous open bijection}.

\begin{thm}{Exercise}\label{ex:bijective-closed-open}
Show that a bijective map between topological spaces is closed if and only if it is open.
\end{thm}

\begin{thm}{Exercise}\label{ex:closed-open-cont}
Give an example of a map $f\:\c{X}\to\c{Y}$ between two topological spaces such that 
\begin{subthm}{}
$f$ is continuous and open, but not closed,
\end{subthm}

\begin{subthm}{}
$f$ is continuous and closed, but not open,
\end{subthm}

\begin{subthm}{}
$f$ is closed and open, but not continuous.
\end{subthm}

\end{thm}

\begin{thm}{Advanced exercise}\label{ex:closed-open-cont-R}
Construct a function $\RR\to\RR$ that is closed and open, but not continuous.
\end{thm}


