In other words, $f\:\c{X}\to \c{Y}$ is an \emph{emebedding} if it is injective and the topology on $\c{X}$ coinsides with pullback topology for $f$.

A function $f\:\c{X}\to \c{Y}$ is called local embedding if for any point $x\in \c{X}$ there is a neighborhood $U\ni x$ such that the restriction $f|_U$ is an embedding.\footnote{The meaning of the word \emph{local} in topology can be described the following way: a space $\c{X}$ is \emph{locally magnificent} if for any point $x\in \c{X}$ and any neighborhood $V\ni x$ there is a smaller neighborhood $W$ (that is, $x\in W\subset V$) which is \emph{magnificent}.}
 
 
 
 
 
 
 
 
 
 
 
 
The following proposition provides a way to construct the weakest topology containing the given collection of sets.

\begin{thm}{Proposition}
Let $\s{Q}$ be arbitrary collection of subsets of a set $\c{X}$.
Consider the set $\s{P}$ of all finite intersections of sets in $\s{Q}$;
that is, $P\in \s{P}$ if there is a finite collection of subsets $Q_1,\dots,Q_n\in \s{Q}$ such that 
\[P=Q_1\cap\dots\cap Q_n.\]
If the collection is empty we assume that $P=\c{X}$.

Denote by $\s{O}$ all possible unions of sets in $\s{P}$;
that is, $O\in \s{O}$ if there is, a (possibly infinite) collection of sets $P_\alpha\in\s{P}$, $\alpha\in\c{I}$ such that 
\[O=\bigcup_{\alpha\in\c{I}}P_\alpha.\]
Then $\s{O}$ is a topology on $\c{X}$.
\end{thm}

\parit{Proof.}
We need to check if the collection $\s{O}$ satisfies the conditions in the definition \ref{def:top-space}.

The conditions \ref{def:top-space:empty} and \ref{def:top-space:u} hold evidently.
It remains to check condition \ref{def:top-space:n};
that is, given two sets $O,O'\in\s{O}$ we need to show that $O_1\cap O_2\in \s{O}$.

Let
\[O=\bigcup_{\alpha\in \c{I}}V_\alpha,\quad O'=\bigcup_{\beta\in \c{J}}V'_\beta,\]
where  $V_\alpha,V'_\beta\in\s{P}$ for any $\alpha\in\c{I}$ and $\beta\in\c{J}$.

Fix $\alpha\in\c{I}$ and $\beta\in\c{J}$.
Since $V_\alpha,\in\s{P}$ it can be presented as an intesection
\[V_\alpha=Q_1\cap \dots \cap Q_n,\]
with $Q_1,\dots,Q_n\in \s{Q}$.
Similarly, 
\[V'_\beta=Q'_1\cap \dots \cap Q'_m,\]
with $Q'_1,\dots,Q'_m\in \s{Q}$.
It follow that 
\begin{align*}
W_{\alpha,\beta}&=V_\alpha\cap V'_\beta=
\\
&=Q_1\cap \dots \cap Q_n\cap Q'_1\cap \dots \cap Q'_m.
\end{align*}
In particular $W_{\alpha,\beta}\in \s{P}$.

Finally note that
\[O\cap O'=\bigcup_{\alpha\in \c{I},\beta\in\c{J}}W_{\alpha,\beta}.\]
Hence $O\cap O'\in \s{O}$.
\qeds
