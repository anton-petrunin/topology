\chapter{Connected spaces}

\section{Definitions}

A subset of topological space is called \index{clopen subset}\emph{clopen} if it is closed and open at the same time.

\begin{thm}{Definition}
A topological space $\c{X}$ is called \index{connected space}\emph{connected} if it has exactly two clopen sets $\emptyset$ and the whole space $\c{X}$.
\end{thm}

According to our definition, \textit{the empty space is not connected}.
Not everyone agrees with this convention.

Suppose $V$ is a \index{proper subset}\emph{proper} clopen subset in a topological space $\c{X}$; that is, $V\ne\emptyset$ and $V\ne\c{X}$.
Note that its complement $W=\c{X}\setminus V$ is also a proper clopen subset.
In particular, there are two open sets $V,W\subset \c{X}$ such that $V\ne\emptyset$, $W\ne\emptyset$, $V\cup W=\c{X}$ and $V\cap W=\emptyset$.

A subset of topological space is called the connected or disconnected if so is the corresponding subspace.
Spelling the notion of subspace we get the following definition.

\begin{thm}{Definition}\label{def:connected-subset}
A subset $A$ of a topological space is called \index{disconnected subset}\emph{disconnected} if it is empty or there are two open sets $V$ and $W$ such that 
\[
V\cap W\cap A=\emptyset,\quad
V\cap A\ne\emptyset,\quad 
W\cap A\ne\emptyset,\quad\text{and}\quad 
V\cup W\supset A.\]

Otherwise, we say that $A$ is \index{connected subset}\emph{connected}.
\end{thm}

A pair of open sets $V$ and $W$ as in the definition will be called \index{open splitting}\emph{open splitting} of $A$.
So we can say that \textit{a nonempty set $A$ is disconnected if and only if it admits an open splitting}.

\begin{thm}{Proposition}\label{prop:image-connected}
Let $f\:\c{X}\to \c{Y}$ be a continuous map between topological spaces.
Suppose $A\subset \c{X}$ is a connected set.
Then the image $f(A)$ is a connected set in $\c{Y}$.
\end{thm}

\parit{Proof.}
We can assume that $B\ne\emptyset$; otherwise the statement is trivial.

Assume that $B=f(A)$ is disconnected.
Choose an open splitting $V$ and $W$ of $B$; that is,
\[V\cap W\cap B=\emptyset,\quad
V\cap B\ne\emptyset,\quad 
W\cap B\ne\emptyset,\quad\text{and}\quad 
V\cup W\supset B.\eqlbl{eq:VWB}\]

Since $f$ is continuous, $V'=f^{-1}(V)$ and $W'=f^{-1}(W)$ are open sets in $\c{X}$.
Note that \ref{eq:VWB} implies that $V'$ and $W'$ is an open splitting of $A$; that is,
\[
V'\cap W'\cap A=\emptyset,\quad
V'\cap A\ne\emptyset,\quad 
W'\cap A\ne\emptyset,\quad\text{and}\quad 
V'\cup W'\supset A.\]
Therefore $A$ is disconnected --- a contradiction.
\qeds

\begin{thm}{Exercise}\label{ex:quotient-connected}
Let $\c{X}$ be a connected space.
Show that the quotient space $\c{X}/{\sim}$ is connected for any equivalence relation $\sim$ on $\c{X}$.
\end{thm}

\begin{thm}{Proposition}\label{prop:union-connect}
Assume $\{A_\alpha\}_{\alpha\in\c{I}}$ is a collection of connected subsets of a topological space.
Suppose that  $\bigcap_\alpha A_\alpha\ne\emptyset$.
Then 
\[A=\bigcup_\alpha A_\alpha\] 
is connected.
\end{thm}

\parit{Proof.}
Assume that $A$ is disconnected; choose its open splitting $V$, $W$.
Since $\bigcap_\alpha A_\alpha\ne\emptyset$, we can fix $p\in \bigcap_\alpha A_\alpha$.
Without loss of generality, we can assume that $p\in V$.

In particular, $V\cap A_\alpha\ne\emptyset$ for any $\alpha$.
Since $A_\alpha$ is connected, we have that $W\cap A_\alpha=\emptyset$ for each $\alpha$; otherwise $V$ and $W$ form an open splitting of $A_\alpha$.
Therefore 
\begin{align*}
W\cap A&=W\cap\biggl(\bigcup_\alpha A_\alpha\biggr)
\\&=\bigcup_\alpha(W\cap A_\alpha)
\\&=\emptyset,
\end{align*}
a contradiction.
\qeds

\begin{thm}{Exercise}\label{ex:A<B<bar-A}
Let $A$ be a connected set in a topological space $\c{X}$.
Suppose that $A\subset B\subset \bar A$.
Show that $B$ is connected.
\end{thm}


\section{Real interval}

\begin{thm}{Proposition}\label{prop:connected[0,1]}
The real interval $[0,1]$ is connected.
\end{thm}

\parit{Proof.}
Assume contrary;
let $V$ and $W$ be an open splitting of $[0,1]$.
Fix a $a_0\in V$ and $b_0\in W$;
without loss of generality, we can assume that $a_0<b_0$.

Let us construct a nested sequence of closed intervals 
\[[a_0,b_0]\supset [a_1,b_1]\supset \dots\]
such that 
\[b_n-a_n=\tfrac1{2^n}(b_0-a_0),\quad a_n\in V,\quad\text{and}\quad b_n\in W
\eqlbl{eq:ab-in-VW}\]
for any $n$.

The construction is recursive.
Assume $[a_{n-1},b_{n-1}]$ is already constructed.
Set $c=\tfrac12\cdot(a_{n-1}+b_{n-1})$.
If $c\in V$,
then set $a_{n}=c$ and $b_{n}=b_{n-1}$;
if $c\in W$, then set $a_{n}=a_{n-1}$ and $b_{n}=c$.
In both cases, \ref{eq:ab-in-VW} holds.

The sequence $a_n$ is nondecreasing and bounded above by $b_0$.
In particular, the sequence $a_n$ converges; denote its limit by $x$.
Since $b_n-a_n=\tfrac1{2^n}\cdot(b_0-a_0)$, the sequence $b_n$ also converges to $x$.
The point $x$ has to belong to $V$ or $W$.
Since both $V$ and $W$ are open, one of them contains $a_n$ and $b_n$ for all large $n$ --- a contradiction.
\qeds

\begin{thm}{Exercise}\label{ex:R-connected}
Show that the real line $\RR$ is a connected space. 
\end{thm}

\section{Connected components}

Let $x$ be a point in a topological space $\c{X}$.
The intersection of all clopen sets containing $x$ is called \index{connected component}\emph{connected component} of $x$.
Note that the space $\c{X}$ is connected if and only if $\c{X}$ is a connected component of some (and therefore any) point in $\c{X}$.

\begin{thm}{Exercise}\label{ex:nonopen-connected-component}
Show that any connected component is a closed set.

Construct an example of topological space $\c{X}$ and a point $x\in \c{X}$ such that the connected component of $x$ is not open.
\end{thm}

\begin{thm}{Exercise}\label{ex:connected-component-disjoint}
Show that two connected components either coincide or disjoint.
\end{thm}

\begin{thm}{Exercise}\label{ex:finite-number-connected-component}
Suppose that a space $\c{X}$ has a finite number of connected components.
Show that each connected component of $\c{X}$ is open.
\end{thm}



\section{Cut points}

Evidently, number of connected components is a topological invariant;
that is, \textit{if two spaces are homeomorphic, then they have the same number of connected components}.
In particular, connected space is not homeomorphic to a disconnected space.

Let us describe a more refined way to apply this observation.
Suppose $\c{X}$ is a connected space, a point $x \in\c{X}$ is called a \index{cut point}\emph{cut point} if removing $x$ from $\c{X}$ produces a disconnected space;
that is, the subset $\c{X}\setminus\{x\}$ is disconnected.

Note that if $f\:\c{X}\to \c{Y}$ is a homeomorphism, then a point $x \in\c{X}$ is a cut point of $\c{X}$ if and only if $y=f(x)$ is a cut point of $\c{Y}$.
Indeed, the restriction of $f$ defines a homeomorphism $\c{X}\setminus\{x\}\to \c{Y}\setminus\{y\}$.
In particular, we get that the spaces $\c{X}\setminus\{x\}$ and $\c{Y}\setminus\{y\}$ have the same number of connected components.

These observations can be used to solve the following exercises.

\begin{thm}{Exercise}\label{ex:S1ne[0,1]}
Show that the circle $\mathbb{S}^1$ is not homeomorphic to the line segment $[0,1]$.
\end{thm}

\begin{thm}{Exercise}\label{ex:R2neR}
Show that the plane $\RR^2$ is not homeomorphic to the real line $\RR$.
\end{thm}

{

\begin{thm}{Exercise}\label{ex:not-homeo}
Show that no two of the following four closed connected sets in the plane are not homeomorphic.
(Each set is a union of four line segments.)

\begin{figure}[!ht]
\centering
\includegraphics{mppics/pic-10}
\end{figure}

\end{thm}

}

{

\begin{wrapfigure}{r}{23 mm}
\vskip-4mm
\centering
\includegraphics{mppics/pic-50}
\end{wrapfigure}

\mbox{\index{Sierpi\'nski gaske}\emph{Sierpi\'nski gasket}} is constructed the following way:
start with a solid equilateral triangle, subdivide it into four smaller congruent equilateral triangles and remove the interior of the central one.
Repeat this procedure recursively for each of the remaining solid triangles.

}


\begin{thm}{Advanced exercise}\label{ex:sierpinski}

\begin{subthm}{ex:sierpinski:connected}
Prove that Sierpi\'nski triangle is connected.
\end{subthm}

\begin{subthm}{ex:sierpinski:homeogroup}
Describe all the homeomorphisms from the Sierpi\'nski triangle to it self.
\end{subthm}

\end{thm}
