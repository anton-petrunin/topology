\chapter{More constructions}

\section{Moving topology by a map}

Recall that in \ref{prebase}, we defined a natural way to move a topology from target space to the source of a map.

Namely, suppose $f\:\c{X}\to\c{Y}$ be a map between two sets.
Assume $\c{Y}$ is equipped with a topology.
Then declare a subset $V\subset\c{X}$ to be open in the \emph{induced topology} for $f$ if there is an open subset $W\subset\c{Y}$ 
such that $V=f^{-1}(W)$.

The following exercise describes an analogous construction that moves a topology from source to target.
It can be solved by checking the conditions in \ref{def:top-space} as we did in \ref{sec:induced-topology}.

\begin{thm}{Exercise}\label{ex:move-topology:pushforward}
Let $f\:\c{X}\to\c{Y}$ be a map between two sets.
Assume $\c{X}$ is equipped with a topology.
Declare a subset $W\subset\c{Y}$ to be open if the subset $V=f^{-1}(W)$ is open in $\c{X}$.
Show that it defines a topology on $\c{X}$.

\end{thm}

The constructed topology on $\c{Y}$ is called \emph{pushforward} topology.

\begin{thm}{Exercise}\label{ex:pullback/pushforward}
Let $f\:\c{X}\to\c{Y}$ be a continuous map.

\begin{subthm}{}
Show that the pullback topology on $\c{X}$ is weaker than its own topology.
\end{subthm}

\begin{subthm}{}
Show that the pushforward topology on $\c{Y}$ is stronger than its own topology.
\end{subthm}

\end{thm}

\begin{thm}{Exercise}\label{ex:category-def-pull-push}
Let $g\:\c{X}\to\c{Y}$ be a continuous map.

\begin{subthm}{}
Suppose $\c{X}$ is equipped with the pullback topology.
Show that a map $f\:\c{W}\to\c{X}$ is continuous if and only if the composition $f\circ g\:\c{W}\to\c{Y}$ is continuous.
\end{subthm}

\begin{subthm}{}
Suppose $\c{Y}$ is equipped with the pushforward topology.
Show that a map $h\:\c{Y}\to\c{Z}$ is continuous if and only if the composition $h\circ f\:\c{X}\to\c{Z}$ is continuous.
\end{subthm}

\end{thm}


The pullback topology is used mostly for injective maps; in this case, it is nearly the same as \emph{induced topology}.
Similarly, pushforward topology is mostly used for surjective maps.
This particular case of the construction is called \emph{quotient topology};
it is discussed in the following section.

\begin{thm}{Exercise}\label{ex:open-closed-pushforward}
Let $f\:\c{X}\to \c{Y}$ be a continuous surjective map.
Assume $f$ is closed or open.
Show that $\c{Y}$ is equipped with the quotient topology.
\end{thm}

\section{Quotient topology}

Let $\sim$ be an \emph{equivalence relation} on a topological space $\c{X}$;
that is, for any $x,y,z\in\c{X}$ the following conditions hold:
\begin{itemize}
 \item $x\sim x$;
 \item if $x\sim y$, then $y\sim x$;
 \item if $x\sim y$ and $y\sim z$, then $x\sim z$.
\end{itemize}

Recall that the set 
\[[x]=\set{y\in \c{X}}{y\sim x}\]
is called the \emph{equivalence class} of $x$.
The set of all equivalence classes in $\c{X}$ will be denoted by $\c{X}/{\sim}$.

Observe that $x\mapsto [x]$ defines a surjective map $\c{X}\to \c{X}/{\sim}$.
The corresponding pushforward topology on $\c{X}/{\sim}$ is called \emph{quotient topology} on $\c{X}/{\sim}$.
By default, $\c{X}/{\sim}$ is equipped with the quotient topology
in this case, it is called \emph{quotient space}.

The following exercise ties equivalence relations with maps.

\begin{thm}{Exercise}\label{ex:eq-relation-f}
Show that an arbitrary map $f\:\c{X}\to \c{Y}$ defines the following equivalence relation on $\c{X}$:
\[x\sim x'\quad\text{if and only if}\quad f(x)=f(x').\]
Moreover,
\[y=f(x)\quad\text{if and only if}\quad [x]=f^{-1}\{f(x)\}.\]
\end{thm}

Given a subset $A$ in a topological space $\c{X}$, the space $\c{X}/A$ is defined as the quotient space $\c{X}/{\sim}$ for the minimal equivalence relation such that $a\sim b$ for any $a,b\in A$.
For example the quotient space $[0,1]/{\sim}$ discussed above can be also denoted by $[0,1]/\{0,1\}$ --- it is the interval $[0,1]$ with identified two-element subset $\{0,1\}$.

\begin{thm}{Exercise}\label{ex:[0,1]/(0,1)}
Describe the quotient space $[0,1]/(0,1)$, where $[0,1]$ and $(0,1)$ are real intervals;
that is, list the points and the open sets of the quotient space.
\end{thm}

\section{Compact-to-Hausdorff trick}

\begin{thm}{Observation}\label{obs:compact-to-hausdorff}
Let $\c{K}$ be a compact space and $\c{Y}$ is Hausdorff.
Then any continuous map $f\:\c{K}\to\c{Y}$ is closed.

If in addition, the map $f$ is onto, then $\c{Y}$ is equipped with the quotient topology induced by $f$. 
\end{thm}

\begin{thm}{Corollary}
A continuous bijection from a compact space to a Hausdorff space is a homeomorphism.
\end{thm}


\parit{Proof of \ref{obs:compact-to-hausdorff}.}
Since $\c{K}$ is compact, any closed subset $Q\subset \c{K}$ is compact as well (\ref{prop:compact-closed}).
Since the image of a compact set is compact we have that $f(Q)$ is a compact subset of $\c{Y}$.
Since $\c{Y}$ is Hausdorff $f(Q)$ is closed.
Hence the first statement follows.

The second statement follows from \ref{ex:open-closed-pushforward}.
\qeds

Recall that $\DD$ denotes the unit disc and $\mathbb{S}^1$ denotes the unit circle;
that is,
\begin{align*}
\DD&=\set{(x,y)\in\RR^2}{x^2+y^2\le 1},
\\
\mathbb{S}^1&=\set{(x,y)\in\RR^2}{x^2+y^2=1}.
\end{align*}

\begin{thm}{Exercise}\label{ex:[0,1]/0=1}
Show that the quotient space $[0,1]/\{0,1\}$ is homeomorphic to $\mathbb{S}^1$.
\end{thm}

\begin{thm}{Exercise}\label{ex:D/S}
Show that the quotient space $\DD/\mathbb{S}^1$ is homeomorphic to the unit sphere 
\[\mathbb{S}^2=\set{(x,y,z)\in\RR^3}{x^2+y^2+z^2=1}.\]
\end{thm}


\section{Orbit spaces}

\begin{thm}{Definition}
Let $\c{X}$ be a topological space and $G$ be a group.
Suppose that $(g,x)\mapsto g\cdot x$ is a map $G\times\c{X}\to \c{X}$ such that
\begin{subthm}{}
$1\cdot x=x$ for any $x\in \c{X}$, here $1$ denotes the identity element of $G$;
\end{subthm}
\begin{subthm}{}
$g\cdot (h\cdot x)=(g\cdot h)\cdot x$ for any $g,h\in G$ and $x\in \c{X}$;%
\footnote{This condition means that the expression $g\cdot h\cdot x$ makes sense; that is, it does not depend on parentheses.}
\end{subthm}
\begin{subthm}{}
for any $g\in G$, the map $x\mapsto g\cdot x$ is continuous.
\end{subthm}
Then we say that $G$ \emph{acts} on $\c{X}$, or $\c{X}$ is a $G$-space.

In this case, the set
\[G\cdot x:=\set{g\cdot x}{g\in G}\]
is called the \emph{$G$-orbit} of $x$ (or, briefly, \emph{orbit}).
\end{thm}

\begin{thm}{Exercise}\label{ex:g-homeo}
Suppose that a group $G$ acts on a topological space~$\c{X}$.
Show that for any $g\in G$, the map $x\mapsto g\cdot x$ defines a homeomorphism $\c{X}\to \c{X}$.
\end{thm}

Suppose that a group $G$ acts on a topological space $\c{X}$.
Set $x\sim y$ if there is $g\in G$ such that $y=g\cdot x$.

Observe that $\sim$ is an equivalence relation on $\c{X}$.
Indeed, 
$x\sim x$ since $x=1\cdot x$.
Further, if $y=g\cdot x$, then 
\[x=1\cdot x=g^{-1}\cdot g\cdot  x=g^{-1}\cdot y;\] since $g^{-1}\in G$ we get that $x\sim y$
$\Longrightarrow$
$y\sim x$.
Finally, suppose $x\sim y$ and $y\sim z$;
that is, $y=g\cdot x$ and $z=h\cdot y$ for some $g,h\in G$.
Then $z=(h\cdot g)\cdot x$;
therefore $x\sim z$.

For the described equivalence relation, the quotient space $\c{X}/{\sim}$ can be also denoted by $\c{X}/G$;
it is called quotient of $\c{X}$ by the action of~$G$.

Note that $[x]=G\cdot x$; that is, the orbit of $x$ coincides with its equivalence class.
By that reason $\c{X}/G$ is also called \emph{orbit space}.

\begin{thm}{Exercise}\label{ex:quotient-map}
Suppose a group $G$ acts on a topological space $\c{X}$ and $f\:\c{X}\to \c{X}/G$ is the quotient map.

\begin{subthm}{}
Show that $f$ is open.
\end{subthm}

\begin{subthm}{}
Assume $G$ is finite.
Show that $f$ is closed.
\end{subthm}

\end{thm}

