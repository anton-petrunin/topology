\chapter{Constructions}

In this chapter we will discuss a few constructions that produce new topological spaces from the given ones.

\section{Induced topology}\label{sec:induced-topology}

Let $A$ be a subset of a topological space $\c{Y}$.
Consider the the so-called \emph{induced topology} on $A$ defined the following way:
a subset $V\subset A$ is open in $A$ if and only if $V=A\cap W$ for an open set $W$ in $\c{Y}$.

Let us check that induced topology is indeed a topology;
in other words, it meets all conditions in \ref{def:top-space}.

First of all the empty set
$\emptyset$ is open since $\emptyset=A\cap \emptyset$.
Further, $A=A\cap \c{Y}$;
therefore $A$ is open in the induced topology.

Assume $\set{V_\alpha}{\alpha\in\c{I}}$ is a collection of open sets in $A$;
that is, for each $V_\alpha$ there is a set $W_\alpha$ which is open in $\c{Y}$ and such that $V_\alpha\z=A\cap W_\alpha$.
Note that
\[\bigcup_\alpha V_\alpha=A\cap\left(\bigcup_\alpha W_\alpha\right).\]
Since the union of $\{W_\alpha\}$ is open in $\c{Y}$, the inion of $\{V_\alpha\}$ is open in the induced topology on $A$.

Assume $V_1$ and $V_2$ are open in $A$; 
that is, $V_1=A\cap W_1$ and  $V_2\z=A\cap W_2$ for some open sets $W_1$ and $W_2$ in $\c{Y}$.
Note that
\[V_1\cap V_2=A\cap(W_1\cap W_2).\]
Since the intersection $W_1\cap W_2$ is open in $\c{Y}$,
the intersection $V_1\cap V_2$ is open in $A$.


A subset $A$ in a topological space $\c{Y}$ equipped with the induced topology is called \emph{subspace} of $\c{Y}$.
It is straightforward to check that this notion agrees with subspace of metric space introduced in \ref{sec:subspaces-metric}.

A function $f\:\c{X}\to \c{Y}$ is called \emph{emebedding} if $f$ defines a homeomorphism from space $\c{X}$ to the subspace $f(\c{X})$ in $\c{Y}$.

\section{Moving topology by a map}

The construction in following exercise moves topology from target space to the source of a map.

\begin{thm}{Exercise}\label{ex:move-topology:pullback}
Let $f\:\c{X}\to\c{Y}$ be a function between two sets.
Assume $\c{Y}$ is equipped with a topology.
Declare a subset $V\subset\c{X}$  to be open if there is an open subset $W\subset\c{Y}$ 
such that $V=f^{-1}(W)$.
Show that it defines a topology on $\c{X}$.
\end{thm}

The constructed topology on $\c{X}$ is called \emph{pullback} topology.
It generalizes the notion of induced topology above.
Namely the induced topology on $A\subset \c{Y}$ can be defined as a pullback topology for the inclusion map $\iota\:A\to \c{Y}$.%
\footnote{The inclusion map $\iota\: A\to \c{X}$ defined as $\iota(a)=a$ for any $a\in A$.}
Indeed, for any $W\subset \c{Y}$ the inverse image $\iota^{-1}(W)$ coincides with the intersection  $V=A\cap W$.

The following exercise describes an analogous construction that moves topology from source to target.
Both exercises can be solved by checking the conditions in \ref{def:top-space} as we did in \ref{sec:induced-topology}.

\begin{thm}{Exercise}\label{ex:move-topology:pushforward}
Let $f\:\c{X}\to\c{Y}$ be a map between two sets.
Assume $\c{X}$ is equipped with a topology.
Declare a subset $W\subset\c{Y}$ to be open if the subset $V=f^{-1}(W)$ is open in $\c{X}$.
Show that it defines a topology on $\c{X}$.

\end{thm}

The constructed topology on $\c{Y}$ is called \emph{pushforward} topology.

\begin{thm}{Exercise}
Let $f\:\c{X}\to\c{Y}$ be a continuous map.

\begin{subthm}{}
Show that the pullback topology on $\c{X}$ is weaker than its own topology.
\end{subthm}

\begin{subthm}{}
Show that the pushforward topology on $\c{Y}$ is stronger than its own topology.
\end{subthm}

\end{thm}

\begin{thm}{Exercise}
Let $g\:\c{X}\to\c{Y}$ be a continuous map.

\begin{subthm}{}
Suppose $\c{X}$ is equipped with the pullback topology.
Show that a map $f\:\c{W}\to\c{X}$ is continuous if and only if the composition $f\circ g\:\c{W}\to\c{Y}$ is continuous.
\end{subthm}

\begin{subthm}{}
Suppose $\c{Y}$ is equipped with the pushforward topology.
Show that a map $h\:\c{Y}\to\c{Z}$ is continuous if and only if the composition $h\circ f\:\c{X}\to\c{Z}$ is continuous.
\end{subthm}

\end{thm}


The pullback topology is used mostly for injective maps; in this case it is nearly the same as \emph{induced topology}.
Similarly pushforward topology is mostly used for surjective maps.
This particular case of the construction is called \emph{quotient toplogy};
it is discussed in the following two sections.

\begin{thm}{Exercise}\label{ex:open-closed-pushforward}
Let $f\:\c{X}\to \c{Y}$ be a continuous surjective map.
Assume $f$ is closed or open.
Show that $\c{Y}$ is equipped with the pushforward.
\end{thm}

\section{Quotient topology}

Let $\sim$ be an \emph{equivalence relation} on a topological space $\c{X}$;
that is, for any $x,y,z\in\c{X}$ the following conditions hold:
\begin{itemize}
 \item $x\sim x$;
 \item if $x\sim y$, then $y\sim x$;
 \item if $x\sim y$ and $y\sim z$, then $x\sim z$.
\end{itemize}

Recall that the set 
\[[x]=\set{y\in \c{X}}{y\sim x}\]
is called \emph{equivalence class} of $x$.
The set of all equivalence classes in $\c{X}$ will be denoted by $\c{X}/\sim$.

The following exercise ties equivalence relations with maps.

\begin{thm}{Exercise}
Show that an arbitrary map $f\:\c{X}\to \c{Y}$ defines the following equivalence relation on $\c{X}$:
\[x\sim x'\quad\text{if and only if}\quad f(x)=f(x').\]
Moreover,
\[y=f(x)\quad\text{if and only if}\quad [x]=f^{-1}\{f(x)\}.\]
\end{thm}

Observe that $x\mapsto [x]$ defines a surjective map $\c{X}\to \c{X}/\sim$.
The corresponding pushforward topology on $\c{X}/\sim$ is called \emph{quotient topology} on $\c{X}/\sim$.
The set $\c{X}/\sim$ with the quotient topology is called \emph{quotient space}.

Intuitively, quotient space is the space obtained by gluing equivalent points together.
For example, consider the \emph{minimal equivalence realtion} on $[0,1]$ such that $0\sim 1$;
that is, $x\sim y$ if and only if one of the following conditions hold $x=y$, or $x=0$ and $y=1$, or $x=1$ and $y=0$.
Then the quotient space $[0,1]/\sim$ is homeomorphic to 
\[\mathbb{S}^1=\set{(x,y)\in\RR^2}{x^2+y^2=1}.\]
A homeomorphism is induced by the map $[0,1]\to \mathbb{S}^1$
\[f(t)=\left(\cos(2\cdot\pi\cdot t),\sin(2\cdot\pi\cdot t)\right).\]
The latter statement can be proved directly from the definition of quotient topology, but soon (\ref{???}) we will have a device that implies this and similar statements efortlesly, so we suggest to wait with a proof of this statement.
