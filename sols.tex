\chapter{Semisolutions}

\begin{multicols}{2}

\parbf{\ref{ex:d1+d2+dinfty}.}
Check all the conditions in Definition~\ref{def:metric-space}.
Further we discuss the triangle inequality --- the remaining conditions are nearly evident.

Let $a=(x_a,y_a)$, $b=(x_b,y_b)$, and $c\z=(x_c,y_c)$.
Set 
\begin{align*}
x_1&=x_b-x_a, 
&
y_1&=y_b-y_a,
\\
x_2&=x_c-x_b,
&
y_2&=y_c-y_b.
\end{align*}

\parit{(a).}
The inequality
$$|a-c|_1\le |a-b|_1+|b-c|_1$$
can be written as 
$$|x_1+x_2|+|y_1+y_2|
\le 
|x_1|+|y_1|+|x_2|+|y_2|.$$
The latter follows since $|x_1+x_2|\le |x_1|+|x_2|$ 
and
$|y_1+y_2|\le |y_1|+|y_2|$.

\parit{(b).}
The inequality
$$|a-c|_2\le |a-b|_2+|b-c|_2\eqlbl{eq:trig-inq-d2}$$
can be written as 
\begin{align*}
&\sqrt{\bigl(x_1+x_2\bigr)^2+\bigl(y_1+y_2\bigr)^2}\le
\\
&\qquad\le
\sqrt{x_1^2+y_1^2}+\sqrt{x_2^2+y_2^2}.
\end{align*}
Take the square of the left and the right-hand sides,
simplify,
take the square again and simplify again.
You should get the following inequality:
$$0
\le 
(x_1\cdot y_2-x_2\cdot y_1)^2,$$
which is equivalent to \ref{eq:trig-inq-d2}
and evidently true.

\parit{(c).}
The inequality
$$|a-c|_\infty\le |a-b|_\infty+|b-c|_\infty$$
can be written as 
$$
\begin{aligned}
&\max\{|x_1+x_2|,|y_1+y_2|\}\le
\\
&\qquad\le 
\max\{|x_1|,|y_1|\}+\max\{|x_2|,|y_2|\}.
\end{aligned}
\eqlbl{eq:max-trig}$$
Without loss of generality, we may assume that 
$$\max\{|x_1+x_2|,|y_1+y_2|\}=|x_1+x_2|.$$
Further,
\begin{align*}
|x_1+x_2|&\le |x_1|+|x_2|\le 
\\
&\le\max\{|x_1|,|y_1|\}+\max\{|x_2|,|y_2|\}.
\end{align*}
Hence \ref{eq:max-trig} follows.

\parbf{\ref{ex:not-a-metric}.}
Check the triangle inequality for $0$, $\tfrac12$, and~$1$.

\parbf{\ref{ex:metric}.}
Check the conditions in \ref{def:metric-space}.

\parbf{\ref{ex:dist-cont}.}
Show that the triangle inequality implies that
$|f(x)-f(y)|<\eps$,
if
$|x-y|_{\c{X}}<\eps$;
make a conclusion.

\parbf{\ref{ex:comp+cont}.}
Fix $x\in \mathcal{X}$ and $y\in\mathcal{Y}$
such that $f(x)=y$.

Fix $\epsilon>0$.
Since $g$ is continuous at $y$, there is a positive value $\delta_1$ such that 
$$d_{\mathcal{Z}}(g(y'),g(y))<\epsilon
\quad
\text{if}
\quad
d_{\mathcal{Y}}(y',y)<\delta_1.$$ 

Since $f$ is continuous at $x$, there is $\delta_2>0$ such that 
$$d_{\mathcal{Y}}(f(x'),f(x))\z<\delta_1
\quad
\text{if}
\quad
d_{\mathcal{X}}(x',x)<\delta_2.$$ 

Since $f(x)=y$, we get that
$$d_{\mathcal{Z}}(h(x'),h(x))<\epsilon
\quad
\text{if}
\quad
d_{\mathcal{X}}(x',x)<\delta_2.$$ 
Hence the result.

\parbf{\ref{ex:isom-cont}.} \ref{SHORT.ex:isom-cont:continuous}
Show that the triangle inequality implies that
$|f(x)-f(y)|_{\c{Y}}<\eps$,
if
$|x-y|_{\c{X}}<\eps$;
make a conclusion.

\parit{\ref{SHORT.ex:isom-cont:injective}.}
Apply \ref{def:metric-space:b}.

\parbf{\ref{ex:to-descrete}.} Show and use that in \ref{def:cont-epsilon-delta} one can take $\delta=1$ for any $\eps>0$.

\parbf{\ref{ex:infinite-inverse-image}.}
Learn about space-filling curves and think.

\parbf{\ref{ex:d1+d2+dinfty-balls}.}
Figure out which is which.

\begin{Figure}
\vskip-0mm
\centering
\includegraphics{mppics/pic-60}
\vskip-0mm
\end{Figure}

\parbf{\ref{ex:r<2R}.}
Apply triangle inequality.
For the second part, describe the balls $\Ball(2,3)$ and $\Ball(0,4)$ in $[0,\infty)$.

\parbf{\ref{ex:def:cont-epsilon-delta=def:cont-balls}.}
Spell the definitions.

\parbf{\ref{ex:ball-is-open}.}
If $y\in \Ball(x,R)$, then $r=R-|x-y|>0$.
Use triangle inequality to show that $\Ball(y,r)\z\subset \Ball(x,R)$.
Make a conclusion.

\parbf{\ref{ex:open-union}.} Apply \ref{def:open}.

\parbf{\ref{ex:open-intersection}.} Apply \ref{def:open}.

\parbf{\ref{ex:union-of-balls}.} The if part follows from \ref{ex:ball-is-open} and \ref{ex:open-union}.
It remains to prove the only-if part.

Let $V$ be an open set. 
By \ref{def:open}, for any $x\in V$ there is $r_x>0$ such that $\Ball(x,r_x)\subset V$.
Observe that
\[V=\bigcup_{x\in V} \Ball(x,r_x).\]

\parbf{\ref{ex:infty-open-intersection}.}
Consider that open segments $(-\eps,\eps)$ for all $\eps>0$ in $\RR$.
Note that 
\[\{0\}=\bigcap_{\eps>0}(-\eps,\eps)\]
and the one-point set $\{0\}$ is not open.

\parbf{\ref{ex:d1+d2+dinfty-open}.}
Show and use that 
\[\Ball(x,r)_1\subset \Ball(x,r)_2\subset \Ball(x,r)_\infty\subset \Ball(x,2\cdot r)_1;\]
here $\Ball(x,r)_1$, $\Ball(x,r)_2$, and $\Ball(x,r)_\infty$ denote the balls in the metrics $|{*}-{*}|_1$, $|{*}-{*}|_2$, and $|{*}-{*}|_\infty$ respectively.

\parbf{\ref{ex:image-of-open}.} 
Look at the image of $\RR$ for the function $x\mapsto |x|$.

\parbf{\ref{ex:unique-lim-metr}.}
Assume the contrary; that is, a sequence $x_1,x_2,\dots$ has two limits $y$ and $z$.
Set $r=|y-z|$.
Note that $\Ball(y,\tfrac r2)$ contains all but finitely many elements of the sequence $x_1,x_2,\dots$;
the same holds for $\Ball(z,\tfrac r2)$.
Observe that $\Ball(y,\tfrac r2)\cap \Ball(z,\tfrac r2)=\emptyset$ and arrive at a contradiction.

\parbf{\ref{ex:continuous-limit}.}
Suppose $f$ is not continuous.
Note that it means that there is a point $x_\infty$ and $\eps>0$ such that there is a point $x_n\in \Ball(x_\infty,\tfrac1n)$ such that $|f(x_n)-f(x_\infty)|>\eps$ in particular, $y_n=f(x_n)$ does not converge to $y_\infty=f(x_\infty)$. 
It proves the if part of the exercise.

To prove the only-if part, suppose that there is a sequence $x_n\to x_\infty$ such that $y_n\not\to y_\infty$ as $n\to \infty$.
Note that in this case we can pass to a subseqence so that $x_n\in \Ball(x_\infty,\tfrac1n)$ and $|y_n-y_\infty|>\eps$ for some fixed $\eps>0$.
From above $f$ is not continuous.

\parbf{\ref{ex:nonclosed-nonopen}.}
Show that the semiopen interval $[0,1)$ is neither open nor closed in $\RR$.

\parbf{\ref{ex:closure-is-closed}.}
Choose a point $z\in \bar{\bar A}$.
It means that, there is a sequence $y_1,y_2,\dots\in\bar A$ such that $y_n\to z$ as $n\to\infty$.
The latter means that for each $y_i$ there is a sequence $x_{i,1}, x_{i,2},\dots\in A$ such that $x_{i,n}\to y_i$ as $n\to\infty$.
Try to choose a sequence of integers $m_n$ such that $x_{n,m_n}\to z$ as $n\to\infty$.
Make a conclusion.

\parbf{\ref{ex:closed-open-complement}.}
Show that $Q$ is closed if $x\in Q$ if and only if $\Ball(x,\eps)\cap Q\ne\emptyset$ for any $\eps>0$.
Show that the latter is equivalent to $y\in V$ if and only if $\Ball(y,\eps)\subset V$ for some $\eps>0$.
Make a conclusion.

\parbf{\ref{ex:cofinite-metrizable}.}
Let $\c{X}$ be an infinite set with cofinite topology.
Show that any two nonempty open sets in $\c{X}$ have nonempty intersection.
Show that the latter does not hold for open balls in a metric space with at least two points.


\parbf{\ref{ex:finite+metrizable}.}
Let $\c{F}$ be a finite metric space.
Observe that there is $\eps>0$ such that $|x-y|>\eps$ for any two distinct points $x,y\in\c{F}$.
It follows that $\{x\}=\Ball(x,\eps)$ for any $x\in\c{F}$;
in particular, each one-point set is open.
By \ref{def:top-space:u}, any set in $\c{F}$ is open.

\parbf{\ref{ex:open-intersection+metrizable}.}
Choose a metric that induce the topology on $\c{X}$.
Let $Q$ be a closed set in $\c{X}$.
Given $\eps>0$, let $W_\eps$ be the union of $\eps$-balls centered at points in $Q$.
Show that $W_\eps$ is open and $Q=\bigcap W_\eps$.

For the second part of the problem, try to find the needed closed set in the connected two-point space.

\parbf{\ref{ex:weaker-top}.}
Choose $W\in \s{W}$.
By assumption, for any $w\in W$ there is $S_w\in \s{S}$ such that $W\supset S_w\ni w$.
Observe that 
\[W=\bigcup_{w\in W}S_w.\]
It follows that $\s{W}\subset \s{S}$;
in other words, any $\s{W}$-open set is $\s{S}$-open.

\parbf{\ref{ex:connect-two-point-maps}}; \ref{SHORT.ex:connect-two-point-maps:R->X}.
Consider the function defined by
\[
f(x)=
\begin{cases}
\quad a &\text{if}\ x< 0,
\\
\quad b &\text{if}\ x\ge 0.
\end{cases}
\]

\parit{\ref{SHORT.ex:connect-two-point-maps:X->R}.} Suppose $f\:\c{X}\to \RR$ is nonconstant;
that is, $f(a)\ne f(b)$.
Note that $W=\RR\setminus\{f(a)\}$ is an open set containing $b$.
Assume $f$ is continuous.
Then $\{b\}=f^{-1}(W)$ is an open set --- a contradiction.

\parbf{\ref{ex:composition-continuous}.}
Apply the definitions.

\parbf{\ref{ex:Rge}}; \ref{SHORT.ex:Rge:top}.Check the conditions in \ref{def:top-space}.

\parit{\ref{SHORT.ex:Rge:nonmetr}.}
Show that that every two nonempty set in $\RR_\ge$ intersect.
Show that the latter statement does not hold in a metric space with at least two points.
Make a conclusion.

\parit{\ref{SHORT.ex:Rge:mono}.}
To do the only-if part check the condition in \ref{def:cont-top}.

To do the if part, suppose $f$ is not nondecreasing;
that is, we can find $x<y$ such that $f(x)>f(y)$.
Note that the inverse image $V=f^{-1}([f(x),\infty))$ contains $x$ but does not contain $y$.
Show that $V$ is not open in $\RR_\ge$, so $f\:\RR_\ge\to\RR_\ge$ is not continuous. 

\parbf{\ref{ex:closed-continuous}.} 
Set $a=f|_A$ and $b=f|_{B}$; choose a closed set $Q\subset \c{Y}$.
By \ref{prop:cont-closed}, $a^{-1}(Q)$ and $b^{-1}(Q)$ are closed.
Note that $a^{-1}(Q)=f^{-1}(Q)\cap A$ and $b^{-1}(Q)=f^{-1}(Q)\cap B$.
Apply \ref{def:closed-properties:n} to show that $f^{-1}(Q)$ is closed.
Further, apply \ref{prop:cont-closed}.

\parbf{\ref{ex:closure-interior-complement}.}
Suppose $V$ is an open subset and $Q$ is its complement.
Recall that $Q$ is closed; see \ref{sec:closed-sets}.
Show and use that $V\subset A$ if and only if $Q\supset B$.

\parbf{\ref{ex:closure-closure}.}
Each of four statements can be deduced by spelling the needed definition.

\parbf{\ref{ex:kuratowski-c}.} Use \ref{ex:closure-closure} together with the definitions of closure and interior.

\parbf{\ref{ex:kuratowski-closed}.}
Observe that $\mathring Q\subset Q$, and apply \ref{ex:kuratowski-c}.

\parbf{\ref{ex:kuratowski-open}.}
Observe that $\bar V\supset V$, and apply \ref{ex:kuratowski-c}.

\parbf{\ref{ex:kuratowski}.}
Try to choose a subset $A$ in $\RR$ so that it meets the following conditions:
\begin{itemize}
\item $A$ and $\RR\setminus A$ contain isolated points,
\item $A$ and $\RR\setminus A$ contain intervals,
\item $A$ and $\RR\setminus A$ are dense in some interval.
\end{itemize}
For the second part, show and use the following 
\[
\bar{\bar A}=\bar A,
\quad
\mathring{\mathring A}=\mathring A,
\quad
\mathring{\bar{\mathring{\bar A}}}=\mathring{\bar A},
\quad\text{and}\quad 
\bar{\mathring{\bar{\mathring A}}}=\bar{\mathring A}.
\]

\parbf{\ref{ex:bry-closed}--\ref{ex:bry<closure}.}
Apply the definitions of boundary and closed set.

\parbf{\ref{ex:3=bry}.}
Read about the Catnor set and think.

\parbf{\ref{ex:bry-nbhd}.}
Let $A$ be a subset of a topological space $\c{X}$.
Denote by $B$ its complement; that is $B\z=\c{X}\setminus A$.
Show and use that the statement is equivalent to the following 
\[\c{X}\setminus\partial A =\mathring A\cup \mathring B.\]

\parbf{\ref{ex:dense-nbhd}.}
Apply the definitions of neighborhood and dense set.

\parbf{\ref{ex:nowhere-dense}.} 
In other words, we need to classify spaces such that any nonempty subset is everywhere dense.
Observe that every point of the space lies in every nonempty open set.
Conclude that the space has concrete topology.
Show that any such space has this property.

\parbf{\ref{ex:def:limit-metric=def:limit-top}.}
Note that any ball $\Ball(x_\infty,\eps)$ is a neighbohood of $x_\infty$.
Observe that it implies the only-if part.

To prove the if part, show and use that for any neighbohood $N$ of $x_\infty$ there is $\eps>0$ such that $\Ball(x_\infty,\eps)\subset N$.

\parbf{\ref{ex:concrete-lim}--\ref{ex:cocountable-top}.}
Apply the definition of the convergence.

\parbf{\ref{ex:bijection-ne-homeo}.}
Let $f\:\c{X}\to\c{Y}$ is a homeomorphism.
Recall that by deinition of inverse, we have $f^{-1}(f(x))=x$ and $f(f^{-1}(y))=y$ for any $x\in \c{X}$ and $y\in\c{Y}$.
It remains to show that the existence of $f^{-1}\:\c{Y}\to\c{X}$ implies that $f$ is a bijection $\c{X}\leftrightarrow\c{Y}$.

For the second part,
try to find such bijection for the subspaces $A=[0,1)\cup \{2\}$ and $B=[0,1]$ of $\RR$.

\parbf{\ref{ex:exp}.}
Observe that $y\mapsto \ln y$ is inverse of $x\mapsto e^x$,
and show that both functions are continuous.
(You can use that differentiable functions are continuous.)

\parbf{\ref{ex:arctan}.} 
Try to build the needed function from $x\mapsto \arctan x$ or $x\mapsto e^{-e^x}$

\parbf{\ref{ex:homeo=eq}.}
Apply the definitions of homeomorphism and \ref{ex:composition-continuous}.

\parbf{\ref{ex:inversion}.}
Learn about inversion and try to apply it.

\parbf{\ref{ex:star-shaped}.}
Let $\Omega$ be an open star-shaped set with respect to the origin.
It is not hard to prove the statement if can be described by the inequality $r<f_n(\theta)$ in the polar $(r,\theta)$-coordinates, were $f_n\:\mathbb{S}^1\to \RR$ is a continuous function
But general star-shaped set, for example one on the diagram is problematic.

\begin{Figure}
\vskip-0mm
\centering
\includegraphics{mppics/pic-71}
\end{Figure}

To do the general case,
show that $\Omega$ can be presented as a union of a nested sequence of open sets $\Omega_0\subset \Omega_1\subset \dots$
such that for each $\Omega_n$ can be described by the inequality $r<f_n(\theta)$ in the polar $(r,\theta)$-coordinates with continuous $f_n\:\mathbb{S}^1\to \RR$.
We can assume that $\Omega_0$ is a round disc around the origin.

Further construct a sequence of homeomorphisms $\phi_n\:\Omega_{n-1}\to\Omega_{n}$ such that 
the compositions $\Phi_n=\phi_n\circ\dots\circ\phi_1\:\Omega_0\to \Omega_n$ stabilises for each $x\in\Omega_0$; that is, $\Phi_n(x)$ is a fixed point for all sufficiently large $n$.
Set 
\[\Phi(x)=\lim_{n\to\infty} \Phi_n(x),\]
and show that $\Phi$ defines the needed homeomorphism $\Omega_0\leftrightarrow \Omega$.

\parbf{\ref{ex:cont-dense}.}
$\Phi_n\:\RR^2\leftrightarrow\RR^2$
Suppose that the sets are $P=\{p_1,p_2,\dots \}$ and $Q=\{q_1,q_2,\dots\}$.
Try to construct a sequence of homeomorphisms $\Phi_n\:\RR^2\leftrightarrow\RR^2$ such that 
$\Phi_n$ converge to a homeomorphism $\Phi\:\RR^2\leftrightarrow\RR^2$ and for any $n$ we have $\Phi_n(\{p_1,\dots,p_n\})\z\subset Q$ and $\Phi_n^{-1}(\{q_1,\dots,q_n\})\z\subset P$.

\parbf{\ref{ex:bijective-closed-open}.}
Apply the definitions.

\parbf{\ref{ex:closed-open-cont}.}
Try the maps between two-point spaces with appropriate topologies.

\parbf{\ref{ex:closed-open-cont-R}.}
Consider a map that vanish on the Cantor set and sends each remaining interval homeomorpically to the whole $\RR$.

\parbf{\ref{ex:cont-product}.} 
Check the following function
\[f(x,y)=
\begin{cases}
0&\text{if}\quad x=0\quad\text{or}\quad y=0,
\\
\tfrac xy &\text{if}\quad 0<x\le y,
\\
\tfrac yx &\text{if}\quad 0<y\le x,
\end{cases}
\]

\parbf{\ref{ex:base};} \textit{only-if part.}
Apply that a base is a collection of open sets.

\parit{If part.} 
Choose an open set $W\subset\c{Y}$.
By \ref{def:base}, 
\[W=\bigcup_\alpha B_\alpha,\]
for some collection $\{B_\alpha\}$ of sets in the base.
Then 
\[f^{-1}(W)=\bigcup_\alpha f^{-1}(B_\alpha).\]
By the assumption $f^{-1}(B_\alpha)$ is open for any $\alpha$;
it remains to apply \ref{def:top-space:u}.

\parbf{\ref{ex:base-nbhd}}; \textit{if part.} 
Choose an open set $N$.
For any $x\in N$ choose an element of base $B_x$ such that $x\in B_x\subset N$.
Observe that 
\[N=\bigcup_{x\in N}B_x.\]

\parit{Only-if part.} Suppose that $\s{B}$ is a base.
Then 
\[N=\bigcup_{\alpha}B_\alpha,\]
where $B_\alpha\in \s{B}$ for each $\alpha$.
Then for any $x\in N$ there is $\alpha$ such that $B_\alpha\ni x$;
in this case, $x\in B_\alpha\subset N$.

\parbf{\ref{ex:prebase};} \textit{only-if part.}
Apply that a prebase is a collection of open sets.

\parit{If part.} Show that for any finite collection of sets $P_1,\dots,P_n$ in the prebase the inverse image 
$f^{-1}(P_1\cap\dots\cap P_n)$
is open.
Further apply \ref{ex:base-nbhd}.

\parbf{\ref{ex:graph}.} To show that the map $F$ is continuous,
apply \ref{ex:prebase} to the prebase described before the exercise.
Further, show and use that projection $G\:(x,f(x))\to x$ is a continuous left inverse;
that is $G(F(x))=x$ for any $x$.

\parbf{\ref{ex:induced-nonmetrizable}.}
Show that \textit{every two disjoint closed sets of a metric space have disjoint open neighborhoods};
that is, for any two closed sets $A$ and $B$ there are open sets $V\supset A$ and $W\supset B$ such that $V\cap W=\emptyset$.
(Topological spaces that share this propery are called \index{normal space}\emph{normal};
so you need to show that \textit{any metrizable space is normal}.)

Observe that arithmetic progression is a closed set in the initial topology.
Construct two disjoint arithmetic progressions that do not admit disjoint open neighborhoods. 

\parbf{\ref{ex:open-open-cover}.}
Use \ref{def:top-space:u} and \ref{def:top-space:n}.

\parbf{\ref{ex:compact-set-subspace}.}
Spell the definitions.

\parbf{\ref{ex:cofinite-compact}.}
We may assume that the space is nonempty; otherwise there is nothing to prove.
Choose a nonempty set $V_0$ from the covering.
Its complement is a vinete set, say $\{x_1,\dots,x_n\}$.
For each $x_i$ choose a set $V_i\ni x_i$ from the covering.
Observe that $\{V_0,\dots,V_0\}$ is a subcover.

\parbf{\ref{ex:unbounded-noncompact}.}
Consider covering of $S$ by intervals $(-c,c)$ for all $c>0$.

\parbf{\ref{ex:closed-compact}.}
Choose a point $s\in\bar S\setminus S$ and consider the cover by intervals $(-\infty, s-\eps)$ and $(s+\eps,+\infty)$ for all $\eps>0$.

\parbf{\ref{ex:noncompact-n}.}
Choose a noncompact space $\c{X}$ and consider topology on the union $\c{X}\cup\{a,b\}$ that includes
$\c{X}\cup\{a,b\}$, $\c{X}\cup\{a\}$, $\c{X}\cup\{b\}$ and all open sets in $\c{X}$.
Observe that the sets $\c{X}\cup\{a\}$ and $\c{X}\cup\{b\}$ are compact, but their intersection is not.

\parbf{\ref{ex:nested-compact}.} Apply the finite intersection property.

\parbf{\ref{ex:S1-compact}.} 
Show that $\mathbb{S}^1$ is an image of closed interval under a continuous map, and apply \ref{prop:comp-image}.

\parbf{\ref{ex:closed-bounded=compact}.}
Apply \ref{prop:compact-closed} and \ref{thm:comp-interval}.

\parbf{\ref{ex:compact-product}.}
Apply \ref{prop:comp-image} to the projections $\c{X}\times\c{Y}\to\c{X}$ and $\c{X}\times\c{Y}\to\c{Y}$.

\parbf{\ref{ex:inscribed-cover-base}.} Apply \ref{ex:base}.

\parbf{\ref{ex:fake-proof(thm:compact-product)}.}
Show by example that the obtained collection 
$\{V_{\alpha_1}\times W_{\alpha_1},\z\dots,V_{\alpha_n}\z\times W_{\alpha_n},V_{\alpha_1'}\z\times W_{\alpha_1'},\z\dots,V_{\alpha_m'}\z\times W_{\alpha_m'}\}$
might not cover the whole $\c{X}\times\c{Y}$.

\parbf{\ref{ex:closed-graph}.}
By \ref{prop:cont-closed}, it is sufficient to show that any closed set $A\subset\c{K}$ has closed inverse image $B=f^{-1}(A)\subset\c{X}$.

Observe that the set $C=\Gamma\cap (\c{X}\times A)$ is closed,
so its complement $U$ can be presented as a union $\bigcup_\alpha V_\alpha\times W_\alpha$.

Suppse $B$ is not closed, choose a point $p\z\in \bar B\backslash B$.
Note that $\{p\}\times \c{K}$ is a compact set in~$U$.
Argue as in \ref{thm:compact-product} to prove that there is an open set $N_p\ni p$ such that $N_p\times \c{K}\subset U$.
Arrive at a contradiction.

\parit{Remark.}
The following function $f\:\RR\to\RR$ has closed graph, but is not continuous:
\[f(x)=
\begin{cases}
\tfrac1x&\text{if}\ x\ne 0,
\\
0&\text{if}\ x= 0.
\end{cases}
\]
It shows that compactness of $\c{K}$ is a necessary assumption.

\parbf{\ref{ex:lebesgue=1}.}
Consider an infinite set of points with discrete metric.

\parbf{\ref{ex:product-sequentially-compact}.}
Suppose that a sequence $x_n$ converges to $x_\infty$ in $\c{X}$ 
and $y_n$ converges to $y_\infty$ in $\c{Y}$.
Show and use that $(x_n,y_n)$ converges to $(x_\infty,y_\infty)$ in $\c{X}\times\c{Y}$ as $n\to\infty$.

\parbf{\ref{ex:compact-complete}.}
By \ref{prop:seq-comp-metr}, any sequence has a converging subseqence;
denote by $x$ the limit of this subsequence.
Show that if the sequence is Cauchy, then it converges to $x$.

\parbf{\ref{ex:hausdorff-unique-limit}.} Arguing by contradiction, assume a sequence has two limits $x$ and $y$.
Since the space is Hausdorff we can choose disjoint neighbohoods $V\ni x$ and $W\ni y$.
Since the sequence converges to $x$, the set $V$ contains all but finitely many elements of the sequence.
The same holds for $W$ --- a contradiction.

\parbf{\ref{ex:hausdorff-diagonal}.}
The set $\Delta$ is closed if and only if its complement $U=(\c{X}\times\c{X})\setminus\Delta$ is open.
Show and use that the latter means that there is a family $\{(V_\alpha,W_\alpha)\}$ of disjoint pairs of open sets in $\c{X}$ such that
\[U=\bigcup_\alpha V_\alpha\times W_\alpha.\]

\parbf{\ref{ex:nonclosed-compact}.}
Look at the subsets of a concrete space.

\parbf{\ref{ex:normal-hausdorff}.} By \ref{thm:hausdorff-compact-t3}, for any $y\in L$ there is a pair of open sets $V_y$, $W_y$ such that $V_y\supset K$ and $W_y\ni y$
such that $V_y\cap W_y=\emptyset$.
Mimic the proof of \ref{thm:hausdorff-compact-t3} using these pairs.

\parbf{\ref{ex:move-topology:pushforward}.} Check the conditions in \ref{def:top-space} directly.

\parbf{\ref{ex:pullback/pushforward}.}

\parbf{\ref{ex:category-def-pull-push}.}

\parbf{\ref{ex:open-closed-pushforward}.}
Since $f$ is continuous, $V=f^{-1}(W)$ is open for any open set $W\subset \c{Y}$.
It remains to show that if $V$ is open so is $W\subset \c{Y}$.

Note that $W=f(V)$.
If $f$ is open, then $W=f(V)$ is open as well.

Set $A=\c{X}\setminus V$ and $B=\c{Y}\setminus W$.
Since $V$ is open $A$ is closed.
Since $f$ is surjective, $B=f(A)$.
Since $f$ is a closed map, $B=f(A)$ is closed as well.
Therefore, $W=\c{Y}\setminus B$ is open.

\parbf{\ref{ex:eq-relation-f}.}
Check the conditions, in the definitions of equivalence relation and equivalence class.

\parbf{\ref{ex:[0,1]/(0,1)}.}
It has tree points $a=[0]$, $b=[\tfrac12]$, and $c=[1]$ and the open sets are 
\[\emptyset,\  \{a,b\},\ \{b,c\},\  \{a,b,c\}.\]

\parbf{\ref{ex:[0,1]/0=1}.}
Apply \ref{obs:compact-to-hausdorff} to the map $[0,1]\to \mathbb{S}^1$ defined by $t\mapsto (\cos (2\cdot\pi\cdot t), \sin(2\cdot\pi\cdot t))$.

\parbf{\ref{ex:D/S}.}
Apply \ref{obs:compact-to-hausdorff} to the map $\DD\to \RR^3$ that is written from polar to spherical coordinates as  
\[(r,\theta)\mapsto (1, \theta, \pi\cdot r).\]

\parbf{\ref{ex:g-homeo}.}
Show and use that $y\mapsto g^{-1}\cdot y$ is inverse of $x\mapsto g\cdot x$.

\parbf{\ref{ex:quotient-map}}
Let $f\:\c{X}\to \c{X}/G$ is the quotient map.
Show that for any set $V\subset \c{X}$ we have
\[f^{-1}\circ f(V)=\bigcup_{g\in G}g\cdot V.\]

\parit{\ref{SHORT.ex:quotient-map:open}.}
Apply this formula and \ref{ex:g-homeo} to show that if $V$ is open then so is $f^{-1}\circ f(V)$.
Finally apply the definition of quotient topology.

\parit{\ref{SHORT.ex:quotient-map:closed}.}
Apply this formula and \ref{ex:g-homeo} to show that if $G$ is finite and $V$ is closed then so is $f^{-1}\circ f(V)$.

\parbf{\ref{ex:quotient-connected}.} Apply \ref{prop:image-connected}.

\parbf{\ref{ex:A<B<bar-A}.}
Show that any open splitting of $B$ splits $A$ as well.

\parbf{\ref{ex:R-connected}.}
Apply \ref{prop:connected[0,1]}, \ref{prop:image-connected}, and \ref{prop:union-connect}.

\parbf{\ref{ex:nonopen-connected-component}.}
Connected component is an intersection of clopen sets; in particular it is closed.

Consider the following subspace of real line $A=\{0,1,\tfrac12,\tfrac13,\dots\}$.
Show that the one-point set $\{0\}$ is a connected component in $A$ and it is not open in $A$. 

\parbf{\ref{ex:connected-component-disjoint}.}
Check that being in one connected component defines an equivalence relation on points of topological space.

\parbf{\ref{ex:finite-number-connected-component}.}
Use \ref{ex:nonopen-connected-component} and \ref{ex:connected-component-disjoint}.

\parbf{\ref{ex:S1ne[0,1]}.}
Show that $\mathbb{S}^1$ has no cut points, but $[0,1]$ has.

\parbf{\ref{ex:R2neR}.}
Show that $\RR^2$ has no cut points, but $\RR$ has.

\parbf{\ref{ex:not-homeo}.}
Count cut points and noncut points for each space.

\parbf{\ref{ex:sierpinski}};
\ref{SHORT.ex:sierpinski:connected}
Let $T_n$ be a the union of all sides of the $3^n$ thriangles after $n^{\text{th}}$ iteration.
Note that the sequence is nested; that is, $T_0\subset T_1\subset \dots{}$
Use induction to show that each $T_i$ is connected.
Conclude that the union $T=T_0\cup T_1\cup\dots$ is connected.
Finally, show that Sierpi\'nski triangle is the closure of $T$ and apply \ref{ex:A<B<bar-A}.

\parit{\ref{SHORT.ex:sierpinski:homeogroup}.}
Denote the Sierpi\'nski triangle by~$\triangle$.

\begin{Figure}
\vskip-0mm
\centering
\includegraphics{mppics/pic-51}
\end{Figure}

Let $\{x,y,z\}$ be a 3-point set in $\triangle$ such that $\triangle \setminus\{x,y,z\}$ has 3 connected components.
Show and use that there is a unique choice for the set $\{x,y,z\}$ and 
it is formed by the midpoints of the original triangle.

\parbf{\ref{ex:convex-path-connected}.}
Let $p$ and $q$ be points in a convex set $F$.
Observe that the linear path 
\[f(t)=(1-t)\cdot p+t\cdot q\]
lies in $F$.

\parbf{\ref{ex:two-point-path-connected}.}
Recall that $\{a\}$ is an open set in $\c{X}$.
Show and use that $f\:[0,1]\to\c{X}$ defined by 
\[f(t)=
\begin{cases}
a&\text{if\ }t<1,
\\
b&\text{if\ }t=1
\end{cases}
\]
is continuous map.

\parbf{\ref{ex:QQ}.}
Show and use that for any rational numbers $a$ and $b$, the line $y=a\cdot x+b$ lies in $A\cup B$.

\parbf{\ref{ex:path-equivalence}.} For \ref{SHORT.ex:path-equivalence:a}, use the constant path $f(t)=x$.
For \ref{SHORT.ex:path-equivalence:b}: if $f$ is a path from $x$ to $y$, then $\bar f$ is from $y$ to $x$.
For \ref{SHORT.ex:path-equivalence:c}, suppose $f$ is a path from $x$ to $y$ and $g$ is a path from $y$ to $z$.
Observe that $f*g$ is a path from $x$ to $z$.

\parbf{\ref{ex:path-connected-component-closed-open}.}

\parbf{\ref{ex:image(path-connected)}.}
Show and use that for any continuous map $\phi$ and any path $f$, the composition $\phi\circ f$ is a path. 

\parbf{\ref{ex:product(path-connected)}.}
Suppose that $f$ and $g$ are paths in $\c{X}$ and $\c{Y}$ respectively.
Show and use that $t\z\mapsto (f(t),g(t))$ is a path in $\c{X}\times\c{Y}$.

\parbf{\ref{ex:locally-path-connected}.} Mimic the proof of \ref{thm:open-connected=path-connected}.

%\parbf{\ref.}

\end{multicols}
