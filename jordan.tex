\chapter{Jordan curve theorem}

\begin{thm}{Theorem}\label{thm:proper-jordan}
Suppose $\Gamma\subset\RR^2$ is a closed set homeomorphic to $\RR$.
Then $\RR^2\backslash \Gamma$ has at least two connected components.
\end{thm}

Note that the assumption that $\Gamma$ is closed is necessary;
indeed a finite open interval $I$ of a line in $\RR^2$ is homeomorphic to $\RR$, but its complement $\RR^2\backslash I$ is connected.

The theorem follows from \ref{prop:c/sc}, \ref{prop:home-gamma-z-axis}, and \ref{ex:z-axis}.

\begin{thm}{Proposition}\label{prop:c/sc}
Suppose $\Gamma\subset\RR^2$ is a closed set such that the complement $X=\RR^2\backslash \Gamma$ is connected.
Let us identify $\RR^2$ with the $(x,y)$-plane in $\RR^3$.
Then the complement $Y=\RR^3\backslash \Gamma$ is simply-connected.
\end{thm}


\parit{Proof.}
Denote by $A$ (respectively $B$) the sets that include $\Gamma$ and the points below (respectively above) $\Gamma$;
that is,
\begin{align*}
A&=\set{(x,y,z)}{(x,y)\in\Gamma\ \text{and}\ z\le 0},
\\
B&=\set{(x,y,z)}{(x,y)\in\Gamma\ \text{and}\ z\ge 0}.
\end{align*}
Consider their complements $V=\RR^3\backslash A$ and $W=\RR^3\backslash B$.
Note that $Y=V\cup W$.

The sets $V$ and $W$ are simply-connected.
Indeed, the horizontal plane $z= 1$ is a deformation retract of $V$;
a retraction can be defined by $(x,y,z)\mapsto (x,y,1)$ and the following homotopy shows that it is homotopic to the identity map: 
\[h_t(x,y,z)=(x,y,(1-t)\cdot z+t).\]
The plane is contractible, in particular simply-connected; therefore so is $V$.
Similarly one proves that $W$ is simply-connected.


Since $X$ is open and connected set in $\RR^2$.
By \ref{thm:open-connected=path-connected}, $X$ is path-connected.
Further, note that $V\cup W=X\times \RR$.
Therefore $V\cup W$ is path-connected as well.

Summarizing, $V$ and $W$ are open simply-connected sets, $Y=V\z\cup W$, and $V\z\cap W$ is path-connected.
Applying \ref{ex:sc-VuW}, we get that $Y$ is simply-connected.
\qeds

\begin{thm}{Proposition}\label{prop:home-gamma-z-axis}
Suppose $\Gamma\subset\RR^2$ is a closed subset homeomorphic to $\RR$.
Then there is a homeomorphism $\RR^3\to \RR^3$ that maps $\Gamma$ the $z$-axis.
\end{thm}


\begin{thm}{Technical lemma}\label{lem:tietze}
Suppose $\Gamma\subset\RR^2$ is a closed subset homeomorphic to $\RR$ and $h\:\RR\to\Gamma$ is a homeomorphism.
Then there is a function $f\:\RR^2\to\RR$ such that 
$f\circ h(t)=t$ for any $t\in \RR$.
\end{thm}

This lemma follows directly from the so-called \emph{Tietze--Urysohn extension theorem},
but we sketch a more elementary proof.

Assume first that $h$ is $L$-Lipschitz;
that is, $|h(t_0)-h(t_1)|\le L\cdot |t_0\z-t_1|$ for any $t_0,t_1\in\RR$.
In this case it is easy to see that the function 
\[f(x)=\sup_{t\in\RR}\{\, t-L\cdot |h(t)-x|\,\}\]
meets the conditions in the lemma.
To do the general case one has to be more inventive with the choice of $f$, but the idea is very the same.

\parit{Sketch of proof.}
Consider the function $\Phi\:\RR\times[0,\infty)\to[0,\infty)$ defined by 
\[\Phi(t,r)\df \sup_{s\in\RR} \{\,|s-t|\cdot (1+r-|h(s)-h(t)|)\,\}.\]
Note that $\Phi$ is continuous.

Moreover, if $r\ge |h(t)-h(s)|$, then
\[|s-t|\le \Phi(t,r)\]
for any $s,t\in\RR$.
It follows that $f\circ h(t)=t$ where  
\[f(p)\df\sup_{t\in\RR}\{\,t-\Phi(t,|p-h(t)|)\,\}.\]
It remains to observe that the function $f\:\RR^2\to\RR$ is continuous.
\qeds

The following proof uses the so-called \emph{Klee trick} which is quite useful in many topological problems.

\parit{Proof of \ref{prop:home-gamma-z-axis}.}
Let $h\:t\mapsto (a(t),b(t))$ be a homeomorphism $\RR\to \Gamma$.
By \ref{lem:tietze}, there is a function $f\:\RR^2\z\to\RR$ such that 
\[f(a(t),b(t))=f\circ h(t)=t\]
for any $t\in \RR$.

Note that the map 
\[F\:(x,y,z)\mapsto (x,y,z+f(x,y))\] is a homeomorphism.
Indeed, this map is continuous and its inverse 
\[F^{-1}\:(x,y,z)\z\mapsto (x,y,z-f(x,y))\]
is continuous as well.

Similarly, the map 
\[G\:(x,y,z)\mapsto (x-a(z),y-b(z),z)\]
is a homeomorphism as well.
Indeed, $G$ is continuous and it has inverse 
\[G^{-1}\:(x,y,z)\mapsto (x+a(z),y+b(z),z)\]
that is continuous as well.

It follows that the composition $G\circ F\:\RR^3\to\RR^3$ is a homeomorphism.
Since $f(a(t),b(t))=t$,
\[G\circ F(a(t),b(t),0)=G(a(t),b(t),t)=(0,0,t).\]
It follows that $G\circ F$ sends $\Gamma$ to the $z$-axis as required.
\qeds

\begin{thm}{Exercise}\label{ex:z-axis}
Show that the complement of the $z$-axis in $\RR^3$ is not simply-connected.
\end{thm}

\begin{thm}{Theorem}
Let $J\subset\mathbb{S}^2$ be a subset homeomorphic to $\mathbb{S}^1$.
Then $\mathbb{S}^2\backslash J$ has at least two connected components. 
\end{thm}

This theorem is a partial case of famous Jordan's theorem;
it is known for simple formulation and annoyingly tricky proofs.
The presented proof is found by Patrick Doyle; 
it uses quite a bit of topology, but is among the shortest.

\parit{Proof.}
Remove a point $p$ from $J$ to get a closed line $\Gamma=J\backslash\{p\}$ in $ \mathbb{S}^2\backslash\{p\}\simeq\RR^2$.
It remains to apply \ref{thm:proper-jordan}.
\qeds
