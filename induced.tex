\chapter{Constructions}

In this chapter we will discuss a few constructions that produce new topological spaces from the given ones.

\section{Induced topology}\label{sec:induced-topology}


\begin{thm}{Proposition}
Let $A$ be a subset of a topological space $\c{Y}$.
Then all subsets $V\subset A$ such that $V=A\cap W$ for some open set $W$ in $\c{Y}$ form a topology on $A$.
\end{thm}

The described topology is called \index{induced topology}\emph{induced topology} on $A$.

A subset $A$ in a topological space $\c{Y}$ equipped with the induced topology is called a \index{subspace}\emph{subspace} of $\c{Y}$.
It is straightforward to check that this notion agrees with the notion introduced in \ref{sec:subspaces-metric};
that is, if $\c{Y}$ is a metric space, then any subset $A\subset\c{Y}$ comes with metric
and the topology defined by this metric coincides with the induced topology on $A$.

A map $f\:\c{X}\to \c{Y}$ is called \index{embedding}\emph{embedding} if $f$ defines a homeomorphism from space $\c{X}$ to the subspace $f(\c{X})$ in $\c{Y}$.

\parit{Proof.}
We need to check the conditions in \ref{def:top-space}.

First, the whole set $A$ and the empty set are included;
indeed, $\emptyset=A\cap \emptyset$ and $A=A\cap \c{Y}$.

Assume $\{V_\alpha\}$ is a collection of open sets in $A$; here $\alpha$ runs in some index set, say $\c{I}$.
In other words, for each $V_\alpha$ there is an open set $W_\alpha\subset\c{Y}$ such that $V_\alpha\z=A\cap W_\alpha$.
Note that
\[\bigcup_\alpha V_\alpha=A\cap\left(\bigcup_\alpha W_\alpha\right).\]
Since the union of $\{W_\alpha\}$ is open in $\c{Y}$ (\ref{def:top-space:u}), the union of $\{V_\alpha\}$ is open in the induced topology on $A$.

Assume $V_1$ and $V_2$ are open in $A$; 
that is, $V_1=A\cap W_1$ and  $V_2\z=A\cap W_2$ for some open sets $W_1,W_2\subset\c{Y}$.
Note that
\[V_1\cap V_2=A\cap(W_1\cap W_2).\]
Since the intersection $W_1\cap W_2$ is open in $\c{Y}$ (\ref{def:top-space:n}),
the intersection $V_1\cap V_2$ is open in $A$.
\qeds

\section{Product space}

Recall that $\c{X}\times\c{Y}$ denotes the set of all pairs $(x,y)$ such that $x\in\c{X}$ and  $y\in\c{Y}$.

Suppose that the sets $\c{X}$ and $\c{Y}$ are equipped with topologies.
Let us construct the \index{product topology}\emph{product topology} on $\c{X}\times\c{Y}$ by declaring that a set is open in $\c{X}\times\c{Y}$ if it can be presented as a union of sets of the following type: $V\times W$ for open sets $V\subset \c{X}$ and $W\subset \c{Y}$.
In other words, a subset $U$ is open in $\c{X}\times\c{Y}$ if and only if there are collections of open sets $V_\alpha\subset \c{X}$ and $W_\alpha\subset \c{Y}$  such that 
\[U=\bigcup_\alpha V_\alpha\times W_\alpha,\]
here $\alpha$ runs in some index set.

By default, we assume that $\c{X}\times\c{Y}$ is equipped with the product topology;
in this case $\c{X}\times\c{Y}$ is called \index{product space}\emph{product space}.

\begin{thm}{Proposition}
The product topology is indeed a topology.
\end{thm}

\parit{Proof.}
Parts \ref{SHORT.def:top-space:empty} and \ref{SHORT.def:top-space:u} in \ref{def:top-space} are evident.
It remains to check \ref{SHORT.def:top-space:n}.
Consider two sets
\[
U=\bigcup_\alpha V_\alpha\times W_\alpha
\quad\text{and}\quad
U'=\bigcup_\beta V'_\beta\times W'_\beta.
\]
where $\alpha$ and $\beta$ run in some index sets, say $\c{I}$ and $\c{J}$ respectively.
We need to show that $U\cap U'$ can be presented as a union of products of open sets;
the latter follows from the next set-theoretical identity
\[U\cap U'=\bigcup_{\alpha,\beta} (V_\alpha\cup V'_\beta)\times (W_\alpha\cup W'_\beta).\eqlbl{eq:nu=un}\]

Checking  \ref{eq:nu=un} is straightforward.
Indeed, $(x,y)\in U\cap U'$ means that $(x,y)\in U$ and $(x,y)\in U'$;
the latter means that $x\in V_\alpha$, $y\in W_\alpha$ and $x\in V'_\beta$, $y\in W'_\beta$ for \textit{some} $\alpha$ and $\beta$.
In other words, $x\in V_\alpha\cap V'_\beta$ and $y\in W_\alpha\cap W'_\beta$ for \textit{some} $\alpha$ and $\beta$;
the latter means that $(x,y)$ belongs to the right-hand side in \ref{eq:nu=un}.
\qeds

Recall that an embedding is defined in \ref{sec:induced-topology}.

\section{Base}

\begin{thm}{Definition}\label{def:base}
A collection $\s{B}$ of open sets in a topological space $\c{X}$ is called its \index{base}\emph{base} if every open set in $\c{X}$ is a union of sets in $\s{B}$.
\end{thm}

The definition is motivated by the fact that \textit{open balls form a base of metric space} (\ref{ex:union-of-balls}).

A base completely defines its topology,
but typically a topology has many different bases.
On metric spaces, for example, the set of all balls with rational radiuses is a base;
another example is the set of all  balls with radiuses smaller than $1$.

In many cases, it is convenient to describe topology by specifying its base.
For example, the product topology on $\c{X}\times\c{Y}$ can be redefined as a \textit{topology with a base formed by all products $V\times W$, where $V$ is open in $\c{X}$, and $W$ is open in $\c{Y}$}.

\begin{thm}{Exercise}\label{ex:base}
Let $\s{B}$ be a base for the topology on $\c{Y}$.
Show that a map $f\: \c{X} \to \c{Y}$ is continuous if and only if $f^{-1}(B)$ is open for any set $B$ in $\s{B}$.

\end{thm}



\begin{thm}{Exercise}\label{ex:base-nbhd}
Let $\s{B}$ be a collection of open sets in a topological space $\c{X}$.
Show that $\s{B}$ is a base in $\c{X}$ if and only if any point $x\in \c{X}$ and any neighborhood $N\ni x$ there is $B\in\s{B}$ such that $x\in B\subset N$.
\end{thm}

\begin{thm}{Proposition}\label{prop:base-any}
Let $\s{B}$ be a set of subsets in some set $\c{X}$.
Show that $\s{B}$ is a base of some topology on $\c{X}$ if and only if it satisfies the following conditions:

\begin{subthm}{prop:base-any:covers}
$\s{B}$ {}\emph{covers} $\c{X}$;
that is, every point $x\in\c{X}$ lies in some set $B\in \s{B}$.
\end{subthm}

\begin{subthm}{prop:base-any:n}
For each pair of sets $B_1, B_2\in \s{B}$ and each point $x \in B_1 \cap B_2$ there exists a set $B
\in \s{B}$ such that $x\in B\subset B_1 \cap B_2$.
\end{subthm}

\end{thm}

\parit{Proof.}
Denote by $\s{O}$ the set of all unions of sets in $\s{B}$.
We need to show that $\s{O}$ is a topology on $\c{X}$.

Evidently, the union of any collection of sets in $\s{O}$ is in $\s{O}$.
Further, $\c{X}$ is in $\s{O}$ by \ref{SHORT.prop:base-any:covers}.
The empty set is in $\s{O}$ since it is a union of the empty collection.

It remins to check \ref{def:top-space:n}; suppose 
\[O=\bigcup_\alpha B_\alpha
\quad\text{and}\quad
O'=\bigcup_\beta B'_\beta,
\]
where $\alpha$ and $\beta$ run in some index sets, and $B_\alpha$, $B'_\beta\in \s{B}$ for any $\alpha$ and~$\beta$.
Then $x\in O\cap O'$ if and only if for some $\alpha$ and $\beta$ we have $x\in B_\alpha$ and $x\in B'_\beta$.
By \ref{SHORT.prop:base-any:n}, we can choose $B\in \s{B}$ so that $x\in B\subset B_\alpha \cap B'_\beta$.
Since $B_\alpha \cap B'_\beta\subset O\cap O'$, it follows that 
\[\textit{for any $x\in O\cap O'$ there is $B_x\in\s{B}$ such that 
$x\in B\subset O\cap O'$.}\]
Observe that 
\[O\cap O'=\bigcup_{x\in O\cap O'} B_x.\]
It follows that $O\cap O'\in\s{O}$ if $O,O'\in\s{O}$.
\qeds

\section{Prebase}\label{prebase}

Suppose $\s{P}$ is a collection of subsets in $\c{X}$ that covers the whole space;
that is, $\c{X}$ is a union of all sets in $\s{P}$.
By \ref{prop:base-any}, the set of all finite intersections of sets in $\s{P}$ is a base for \textit{some} topology on $\c{X}$.
The set $\s{P}$ is called \index{prebase}\emph{prebase} for this topology (also known as \index{subbase}\emph{subbase}).

\begin{thm}{Exercise}\label{ex:prebase}
Let $\s{P}$ be a prebase for the topology on $\c{Y}$.
Show that a map $f\: \c{X} \to \c{Y}$ is continuous if and only if $f^{-1}(P)$ is open for any set $P$ in $\s{P}$.

\end{thm}

There are almost no restrictions on prebase --- we may start with any collection $\s{P}$ of subsets of $\c{X}$ that covers the whole $\c{X}$ and define a topology by declaring that \textit{$\s{P}$ is a prebase for the topology}.
It defines the weakest topology on $\c{X}$ such that every set of $\s{P}$ is open.

For example, the product topology on $\c{X}\times\c{Y}$ can be redefined as a \textit{topology with prebase formed by all products $\c{X}\times W$ and $V\times \c{Y}$, where $V$ is open in $\c{X}$ and $W$ is open in $\c{Y}$}.


\begin{thm}{Exercise}\label{ex:graph}
Given a map $f\:\c{X}\z\to \c{Y}$, consider the map $F\:\c{X}\z\to \c{X}\times\c{Y}$ defined by $F\:x\mapsto (x,f(x))$.
Show that $f$ is continuous if and only if $F$ is an embedding.
\end{thm}

\section{Family of maps}\label{induced-family}

Note that the product topology on $\c{X}\times\c{Y}$ is the weakest topology such that the following two projections are continuous:
$\c{X}\times\c{Y}\to \c{X}$ and $\c{X}\times\c{Y}\to \c{Y}$ defined by $(x,y)\mapsto x$ and $(x,y)\mapsto y$ respectively.

Indeed, these projections are continuous if the inverse images of all opens sets in $\c{X}$ and $\c{Y}$ are open in $\c{X}\times\c{Y}$.
In other words the topology on $\c{X}\times\c{Y}$ must contain all sets of the form $V\times \c{Y}$ and $\c{X}\times W$ for open sets $V\subset \c{X}$ and $W\subset \c{Y}$.
These sets form a prebase in $\c{X}\times\c{Y}$ and its topology is the weakest topology  that contains these sets.

More generally, given a collection of maps $f_\alpha\:S\to \c{Y}_\alpha$ from a set $S$ to topological spaces $\c{Y}_\alpha$, we can introduce topology on $S$ by stating that the inverse images $f^{-1}_\alpha(W_\alpha)$ for open sets $W_\alpha\subset\c{Y}_\alpha$ form its prebase.
It defines topology on $S$ \index{induced topology}\emph{induced} by the maps $f_\alpha$;
it is weakest topology on $S$ that makes all maps $f_\alpha$ to be continuous.

This construction produces a topology on the source space;
by that reason it is also called \index{initial topology}\emph{initial topology}.
In \ref{sec:final-topology} we will discuss final topology --- an analogous construction that moves to topology from source to target of a map.

For example, this construction can be used to define topology on infinite product of spaces, as the induced topology for all its projections.
This topology is called \index{product topology}\emph{product topology}, or \index{Tychonoff topology}\emph{Tychonoff topology}.

\begin{thm}{Advanced exericse}\label{ex:induced-nonmetrizable}
Let $\s{O}$ be the topology on $\RR$ induced by the maps $x\mapsto (\cos(t\cdot x),\sin(t\cdot x))$ for all $t\in \RR$.
Show that the space $(\RR,\s{O})$ is not metrizable.
\end{thm}

