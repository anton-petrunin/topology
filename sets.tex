\chapter{Subsets}

\section{Closed sets}

Let $\c{X}$ be a topological space. 

A set $K\subset \c{X}$ is called \emph{closed} if its complement $\c{X}\backslash K$ is open.

From the definition of topological spaces the following properties of closed sets follow.

\begin{thm}{Proposition}\label{prop:closed-properties}
Let $\c{X}$ be a topological space. 
\begin{enumerate}[(i)] 
\item The empty set and $\c{X}$ are closed.
\item The intersection of any collection of closed sets is a closed set.
That is, if $K_\alpha$ is open for any $\alpha$ in the index set $\c{I}$,
then the set
\[Q=\bigcap_{\alpha\in \c{I}} K_\alpha
=
\set{x\in \c{X}}{x\in K_\alpha\quad\text{for any}\quad\alpha\in \c{I}}\]
is closed
\item\label{def:closed-properties:n} 
The union of two closed sets (or any finite collection of closed sets) is closed.  
That is,  if $K_1$ and $K_2$ are closed, then the union $Q=K_1 \cup K_2$ is closed. 
\end{enumerate}
\end{thm}

The definitions of open and closed sets are mirror-symmetric to each other.
There is no particular reason why we define topological space using open sets --- we could use closed sets instead.%
\footnote{In fact, closed sets were considered before open sets --- the former were introdiced by Georg Cantor in 1884, and the latter by René Baire in 1899.}

Sometimes it is easier to use closed sets; 
for example, the cofinite topology can be defined by declaring that the whole space and all its finite sets are closed.

The following proposition is completely analogous to the original definition of continuous functions via open sets (\ref{def:cont-top}). 

\begin{thm}{Proposition}\label{prop:cont-closed}
Let $\c{X}$ and $\c{Y}$ be topological spaces.
A function $f\:\c{X}\to\c{Y}$ is continuous if and only if any closed set $K$ has closed inverse image $f^{-1}(K)$.
\end{thm}


\parit{Proof.}
In the proof, we will use following set-theoretical identity.

Suppose $A\subset \c{Y}$ and $B=\c{Y}\backslash A$ (equvalently  $A=\c{Y}\backslash B$).
Then
$$f^{-1}(B)
=
\c{X}\backslash f^{-1}(A)\eqlbl{eq:inverse-image-complement}$$
for any function $f\:\c{X}\to\c{Y}$.
This identity is tautological,
to prove it observe that both sides can be spelled as 
\[\set{x\in\c{X}}{f(x)\notin A}.\]

\parit{``Only-if'' part.}
Let $B\subset \c{Y}$ be a closed set.
Then $A=\c{Y}\backslash B$ is open.
Since $f$ is continuous, 
$f^{-1}(A)$ is open.
By \ref{eq:inverse-image-complement},
$f^{-1}(B)$ is the complement of $f^{-1}(A)$ in $\c{X}$.
Hence $f^{-1}(B)$ is closed.

\parit{``If'' part.}
Fix an open set $B$, its complement $A=\c{Y}\backslash B$ is closed.
Therefore $f^{-1}(A)$ is closed.
By \ref{eq:inverse-image-complement},
$f^{-1}(B)$ is a complement of $f^{-1}(A)$  in $\c{X}$. 
Hence $f^{-1}(B)$ is open.

The statement follows since $B$ is an arbitrary open set.
\qeds

\section{Interior and closure}

Let $A$ be an arbitrary subset in a topological space $\c{X}$.
The union of all open subsets of $A$ is called the \emph{interior} of $A$ and denoted as $\mathring A$.

Note that $\mathring A$ is open.
Indeed, it is defined as a union of open sets and such union has to be open by definition of topology (\ref{def:top-space}).
So we can say that $\mathring A$ is the \emph{maximal} open set in $A$, 
as any open subset of $A$ lies in~$\mathring A$.

In a similar fashion, we define closure.
The intersection of all closed subsets containing $A$ is called the \emph{closure} of $A$ and denoted as $\bar A$.

The set $\bar A$ is closed.
Indeed, it is defined as an intersection of closed sets and such intersection has to be closed by Proposition~\ref{prop:closed-properties}.
In other words, $\bar A$ is the minimal closed set that contains $A$, 
as any closed subset of $A$ contains $\bar A$.

\begin{thm}{Exercise}
Assume $A$ is a subset of a topological space $\c{X}$;
consider its complement $B=\c{X}\backslash A$.
Show that 
\[\bar B=\c{X}\backslash \mathring A.\]
\end{thm}

\begin{thm}{Exercise}
Show that the following holds for any set $A$ of a topological space:

\begin{subthm}{}
$\mathring A\subset A\subset \bar A$
\end{subthm}

\begin{subthm}{}
$\bar{\bar A}=\bar A$
\end{subthm}

\begin{subthm}{}
$\mathring{\mathring A}=\mathring A$
\end{subthm}

\end{thm}

\begin{thm}{Exercise}

\begin{subthm}{}
Give an example of a topological space $\c{X}$ with a closed subset $Q$ such that
 \[\bar{\mathring Q}\ne Q.\]
\end{subthm}

\begin{subthm}{}
Show that 
\[\mathring{\bar{\mathring Q}}=\mathring Q\]
for any closed set $Q$.
\end{subthm}

\begin{subthm}{}
Give an example of a topological space $\c{X}$ with an open subset $V$ such that
 \[\mathring{\bar V}\ne V.\]
\end{subthm}

\begin{subthm}{}
Show that 
\[\bar{\mathring{\bar V}}=\bar V\]
for any open set $V$.
\end{subthm}

\begin{subthm}{}
 Give an example of a topological space $\c{X}$ with a subset $A$ such that all the following 7 subsets are distinct:
\[\bar{\mathring{\bar A}},\ \mathring{\bar A},\ \bar A,\  A,\  \mathring A,\ \bar{\mathring A},\ \mathring{\bar{\mathring A}}.\]
\end{subthm}

\end{thm}

\section{Boundary}

Let $A$ be an arbitrary subset in a topological space $\c{X}$.
The \emph{boundary} of $A$ (briefly $\partial A$) is defined as the complement 
\[\partial A=\bar A\backslash \mathring A.\]

\begin{thm}{Exercise}
Show that the boundary of any set is closed.
\end{thm}

\begin{thm}{Exercise}
Show that the set $A$ is closed if and only if $\partial A\subset A$.
\end{thm}

\begin{thm}{Advanced exercise}
Find three disjoint open sets on the real line 
that have the same nonempty boundary.
\end{thm}

\section{Neighborhoods}

Let $x$ be a point in a topological space $\c{X}$.
A \emph{neighborhood} of $x$ is any open set $U$ containing $x$.
In topology, neighborhoods often replace the notion of ball (the latter can be used only in metric spaces).  


\begin{thm}{Exercise}
Let $A$ be a set in a topological space $\c{X}$.
Show that $x\in \partial A$ if and only if any neighborhood of $x$ contains points in $A$ and its complement $\c{X}\backslash A$. 
\end{thm}

Let $A$ and $B$ be subsets of a topological space $\c{X}$.
The set $A$ is said to be a \emph{dense in}  $B$ if $\bar A\supset B$.

\begin{thm}{Exercise} Show that $A$ is dense in $B$ if and only any neighborhood of any point in $B$ intersects $A$.
\end{thm}

