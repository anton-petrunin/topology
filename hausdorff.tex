\chapter{Hausdorff spaces}

\section{Definition}

\begin{thm}{Definition} 
A topological space $\c{X}$ is called \emph{Hausdorff} if for each pair of distinct points
$x, y \in \c{X}$ there are disjoint neighborhoods $V\ni x$ and $W\ni y$.
\end{thm}

Recall that a sequence $x_n$ of points in a topological space $\c{X}$ converges to a point $x\in  \c{X}$
if for any neighborhood $V\ni x$ we have 
$x_n\in V$ for all, but finitely many $n$.

Note that in general, a sequence of points in a topological space might have different limits.
For example, consider the real line with the cofinite topology and a sequence $x_1,x_2,\dots$ such that $x_m\ne x_n$ for $m\ne n$.
Note that the sequence converges to \textit{every} $x\in \RR$.
Indeed, a complement of any neighborhood $V\ni x$ is a finite set;
therefore $x_n\in V$ for all, but finitely many indexes $n$.

\begin{thm}{Exercise}\label{ex:hausdorff-unique-limit}
Show that any converging sequence in Hausdorff space has a unique limit.
\end{thm}


\begin{thm}{Exercise}\label{ex:hausdorff-diagonal}
Show that the topological space $\c{X}$ is Hausdorff
if and only if the diagonal 
\[\Delta=\{\,(x,x)\in \c{X}\times\c{X}\,\}\]
is a closed set in the product space $\c{X}\times\c{X}$.
\end{thm}

\section{Observations}

\begin{thm}{Observation}\label{obs:one-point=compact}
Any one-point set in a Hausdorff space is closed. 
\end{thm}

If every one-point space in a topological space is closed
then the space is called \emph{$T_1$-space} or sometimes \emph{Tikhonov space}.
Therefore the observation above states that any Hausdorff space is $T_1$.

\parit{Proof.}
Choose a point $x$ in a Hausdorff space $\c{X}$.
By definition of Hausdorff space, given a point $y\ne x$,
there are disjoint open sets $V_y\ni x$ and $W_y\ni y$.
In particular $W_y\not\ni x$.

Note that 
\[\c{X}\setminus\{x\}=\bigcup_{y\ne x}W_y.\]
It follows that $\c{X}\setminus\{x\}$ is open, and therefore $\{x\}$ is closed.
\qeds

\begin{thm}{Observation}
Any metrizable space is Hausdorff.
\end{thm}

\parit{Proof.}
Assume that topology on the space $\c{X}$ is induced by a metric $|{*}-{*}|$.

If the points $x,y\in\c{X}$ are distinct then $|x-y|>0$.
By triangle inequality 
$\Ball(x,\tfrac r2)\cap \Ball(y,\tfrac r2)=\emptyset$.
Hence the statement follows
\qeds


\begin{thm}{Observation}
Any subspace of Hausdorff space is Hausdorff.
\end{thm}

\parit{Proof.}
Choose two points $x,y$ in a subspace $A$ of a Hausdorff space $\c{X}$.
Since $\c{X}$ is Hausdorff, we can choose neighborhoods $V\ni x$ ad $W\ni y$ such that $V\cap W=\emptyset$.
Then $A\cap V$ and $A\cap W$ are neighborhoods of $x$ and $y$ in $A$.
Clearly, 
\[(A\cap V)\cap(A\cap W)\subset V\cap W=\emptyset.\]
Whence the observation follows.
\qeds


\section{Games with compactness}

\begin{thm}{Proposition}\label{prop:hausdorff-compact-closed}
Any compact  subset of a Hausdorff space is closed. 
\end{thm}

Noote that any one-point set is compact.
Therefore the proposition generalizes Observation~\ref{obs:one-point=compact}.
The proof is similar but requires an extra step.
It is instructive to solve the following exercise before reading the proof.

\begin{thm}{Exercise}\label{ex:nonclosed-compact}
Describe a topological space $\c{X}$ with a nonclosed, but compact subset $K$. 
\end{thm}

The proof of proposition is beased on the following theorem.

\begin{thm}{Theorem}\label{thm:hausdorff-compact-t3}
Let $\c{X}$ be a Hausdorff space and $K\subset X$ be a compact subset.
Then for any point $y\notin K$ there are open sets $V\supset K$ and $W\ni y$
such that $V\cap W=\emptyset$
\end{thm}

\parit{Proof of \ref{prop:hausdorff-compact-closed} modulo \ref{thm:hausdorff-compact-t3}.}
For $y\notin K$, let us denote by $W_y$ the open set provided by \ref{thm:hausdorff-compact-t3};
in particular, $W_y\ni y$ and $W_y\cap K=\emptyset$.
Note that 
\[\c{X}\setminus K=\bigcup_{y\notin K} W_y.\]
It follows that $\c{X}\setminus K$ is open; therefore, $K$ is closed.
\qeds


\parit{Proof.}
By definition of Hausdorff space,
for any point $x\in K$ there is a pair of disjoint openset $V_x\ni x$ and $W_x\ni y$.
Note that sets $\{V_x\}_{x\in K}$ forms cover of $K$.
Since $K$ is compact we can pass to a finite subcover $\{V_{x_1},\dots , V_{x_n}\}$.
Set 
\begin{align*}
V&=V_{x_1}\cup\dots \cup V_{x_n}
\\
W&=W_{x_1}\cap\dots \cap W_{x_n}
\end{align*}
It remains to observe that $V$ and $W$ are open, $y\in W$, $K\subset V$, and 
\[V\cap W\subset \bigcup_i(V_{x_i}\cap W_{x_i})=\emptyset.\]
\qedsf

\begin{thm}{Exercise}\label{ex:normal-hausdorff}
Let $\c{X}$ be a Hausdorff space and $K,L\subset X$ be two compact subsets.
Assume $K\cap L=\emptyset$, show that there are open sets $V\supset K$ and $W\supset L$
such that $V\cap W=\emptyset$
\end{thm}




