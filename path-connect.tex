\chapter{Path-connected spaces}

\section{Definitions}

Let $\c{X}$ be a topological space.
A continuous map $f\:[0,1]\to \c{X}$ is called \emph{path}.
If $x=f(0)$ and $y=f(1)$ we say that $f$ is a path from $x$ to $y$.

A space $\c{X}$ is called \emph{path-connected} if it is nonempty and any two points in $\c{X}$ can be connected by a path;
that is, for any $x,y\in \c{X}$ there is a path $f$ from $x$ to $y$.

\begin{thm}{Exericise}
Show that any convex set in a Euclidean space is path-connected. 
\end{thm}

\begin{thm}{Theorem}\label{thm:conncted/path-connected}
Any path-connected space is connected;
the converse does not hold.
\end{thm}

\parit{Proof; main part.}
Let $\c{X}$ be a path-connected space.

By Proposition~\ref{prop:connected[0,1]}, the unit interval $[0,1]$ is connected.
By Proposition~\ref{prop:image-connected}, for any path $f\:[0,1]\to \c{X}$ the image $f([0,1])$ is connected.

Fix $x\in \c{X}$. 
Since $\c{X}$ is path-connected, 
\[\c{X}=\bigcup_f f([0,1]),\]
where the union is taken for all paths $f$ starting from $x$.
By Proposition~\ref{prop:union-connect}, $\c{X}$ is connected.

\parit{Second part.}
It remains to present an example of a space that connected, but not path-connected.

Denote by $I$ the closed line segment from $(0,0)$ to $(1,0)$ in the plane.
Further, denote by $J_n$ the closed line segment from $(\tfrac1n,0)$ to $(\tfrac1n,1)$.
Consider the union of all these segments
\[W=I\cup J_1\cup J_2 \cup \dots\]
and set 
\[W'=W\cup\{y\},\]
where $y=(0,1)$.
The space $W'$ is called \emph{flea and comb};
the set $W$ is called \emph{comb},
and the point $y$ is called \emph{flea}.

\begin{wrapfigure}{r}{33mm}
\vskip-4mm
\includegraphics{mppics/pic-20}
\end{wrapfigure}

Note that $W\subset W'\subset \bar W$.
Therefore, by \ref{ex:A<B<bar-A}, $W'$ is connected.

It remains to show that $W'$ is not path-connected.
Assume the contrary. 
Let $f$ be a path from $x=(0,0)$ to $y=(0,1)$.

Note that $f^{-1}(\{y\})$ is closed subset of compact space $[0,1]$.
Therefore $f^{-1}(\{y\})$ is compact.
In particular, the set $f^{-1}(\{y\})$ has the minimal element,
denote it by $b$.
Note that $b>0$; so $f(b)=y$ and $f(t)\ne y$ for any $t<b$.

Choose positive $\eps<1$.
Since $f$ is continuous, there is $a<b$ such that $|f(t)-y|<\eps$ for any $t\in [a,b]$.
Note that $f(a)\in J_n$ for some $n$.
Denote by $N$ the intersection of $\eps$-neighborhood of $y$ with the comb.
note that the intersection of $J_n$ with $\eps$-neighborhood of $y$ forms a connected component of $N$.
It follows that $f(t)\in J_n$ for any $t\in [a,b]$;
in particular, $f(b)\ne y$ --- a contradiction.
\qeds

\section{Operations on paths}\label{sec:Operations on paths}

Given a path $f\:[0,1]\to \c{X}$ one can consider the \emph{time-reversed} path $\bar f$.
Namely, 
\[\bar f(t)=f(1-t).\]
Note that $\bar f$ is continuous since $f$ is.

Let $f$ and $h$ be paths in the topological space $\c{X}$.
If $f(1)=h(0)$ we can join these two paths into one $g\:[0,1]\to \c{X}$ defined as 
\[g(t)=
\left[
\begin{aligned}
f(2\cdot t)&&\text{if}&\ t\le \tfrac12
\\
h(2\cdot t-1)&&\text{if}&\ t\ge \tfrac12
\end{aligned}
\right.
\]
The path $g$ is called the \emph{product} (or \emph{concatenation}) of paths $f$ and $h$, briefly it is denoted as $g=f * h$.

By 
In order to show that $f * h$ is indeed a path, we need to check that the defined map $f * h\:[0,1]\to \c{X}$ is continuous.

Indeed, fix a closed set $C\subset \c{X}$,
assume $E=(f * h)^{-1}(C)\subset [0,1]$.
Since $f$ is continuous, we get that $E\cap [0,\tfrac12]$ is closed.
The same way, since $h$ is continuous, we get that $E\cap [\tfrac12,1]$ is closed.
Since 
\[E=(E\cap [0,\tfrac12])\cup (E\cap [\tfrac12,1]),\] 
it follows that $E$ is a closed subset in $[0,1]$.
That is, the inverse image of any closed set in $\c{X}$ is closed in $[0,1]$ --- by Proposition~\ref{prop:cont-closed}, $f * h\:[0,1]\to \c{X}$ is continuous.


Consider the relation $\sim$ on the set of points of topological space defined as $x\sim y$ if there is a path from $x$ to $y$.

\begin{thm}{Exercise}
Show that $\sim$ is an equivalence relation;
that is, for any points $x$, $y$, and $z$ in a topological space we have

\begin{subthm}{}
$x\sim x$.
\end{subthm}

\begin{subthm}{}
If $x\sim y$, then $y\sim x$
\end{subthm}

\begin{subthm}{}
If $x\sim y$ and $y\sim z$, then $x\sim z$.
\end{subthm}

\end{thm}

The equivalence class of point $x$ for the equivalence relation $\sim$ is called \emph{path-connected component} of $x$. 

\begin{thm}{Exercise}
Assume every path-connected component in a topological space $\c{X}$ is closed.
Show that $\c{X}$ is connected if and only if $\c{X}$ is path-connected.
\end{thm}

\begin{thm}{Exercise}
Show that image of path-connected set under a continuous map is path-connected.
\end{thm}

\begin{thm}{Exercise}
Show that the product of path-connected spaces is path-connected.
\end{thm}

\section{Open sets of Euclidean space}

The following theorem provides a class of topological space for which connectedness implies path-connectedness.

\begin{thm}{Theorem}\label{thm:open-connected=path-connected}
An open set in a Euclidean space $\RR^n$ is path-connected if and only if it is connected.
\end{thm}

\parit{Proof.}
The only-if part follows from \ref{thm:conncted/path-connected};
it remains to prove the if part.

Let $\Omega\subset\RR^n$ be an open subset.
Choose a point $p\in\Omega$; denote by $P\subset \Omega$ the path-connected component of $p$.

Let us show that for any point $q\in \Omega$ there is $\eps>0$ such that either $\Ball(q,\eps)\subset P$, or $\Ball(q,\eps)\cap P=\emptyset$.

Indeed, since $\Omega$ is open, we can choose $\eps>0$ such that $\Ball(q,\eps)\z\subset\Omega$.
Note that $\Ball(q,\eps)$ is convex.
Therefore if $\Ball(q,\eps)\cap P=\emptyset$, then $\Ball(q,\eps)\subset P$.

It follows that $P$ and its complement $\Omega\setminus P$ are open.
Since $\Omega$ is connected, we get that $\Omega\setminus P=\emptyset$ --- hence the result.
\qeds

A topological space $\c{X}$ is called \emph{locally path-connected} if any point $p\in\c{X}$ and its neighborhood $V$ there is a path-connected open set $W$ such that $V\supset W\ni p$. 
Note that Euclidean spaceis locally path-connected;
it follows since any open ball in a Euclidean space is path-connected.
Therefore the following exercise generalizes the theorem above.

\begin{thm}{Exercise}
Show that an open set in a locally path-connected space is path-connected if and only if it is connected.
\end{thm}

