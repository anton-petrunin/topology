\chapter{Homotopy}


\section{Definitions}

Assume $f_t\:\c{X}\to \c{Y}$ be a one-parameter family of maps, $t\in [0,1]$.
If the map $[0,1]\times \c{X}\to \c{Y}$ that is defined as $(t,x)\mapsto f_t(x)$ is continuous, then $f_t$ is called a \emph{homotopy} of maps from $\c{X}$ to $\c{Y}$.

If $f_t(a)$ is independent of $t$ for any point $a$ in some subset $A\subset \c{X}$, then we say that  $f_t$ is a \emph{homotopy relative to $A$}. 

Two maps $g,h\:\c{X}\to \c{Y}$ are called \emph{homotopic} (briefly $g\sim h$)
if there is a homotopy $f_t\:\c{X}\to \c{Y}$ such that $g=f_0$ and $h=f_1$.

Two maps $g,h\:\c{X}\to \c{Y}$ are called \emph{homotopic relative to} $A$, briefly $g\sim h\rel A$
if there is a homotopy $f_t\:\c{X}\to \c{Y}$ relative to $A$ such that $g=f_0$ and $h=f_1$.

\begin{thm}{Exercise}\label{ex:hom-eq}
Show that ``$\sim$''  defines equivalence relations on the set of continuous maps between given topological spaces.
That is, for any continuous maps $f,g,h\:\c{X}\to \c{Y}$, we have 
\begin{itemize}
\item $f\sim f$, 
\item if $f\sim h$ then $h\sim f$, and
\item if $f\sim g$ and $g\sim h$ then $f\sim h$.
\end{itemize}
Do the same for and ``$\sim\rel A$''.
\end{thm}



\section{Contractible spaces}

A topological space $\c{X}$ is called \emph{contactable} if the identity map $\id_\c{X}\:\c{X}\z\to \c{X}$ is homotopic to a constant map;
that is there is a homotopy $h_t\:\c{X}\to\c{X}$ such that $h_0(x)=x$ and $h_1(x)=p$ for some fixed point $p\in\c{X}$ and any $t$.

\begin{thm}{Exercise}
Show that any convex subset of the Euclidean space is contractible.
\end{thm}


\begin{thm}{Exercise}
Show that any contractible space is path-connected.
\end{thm}

Later we will show that the circle $\mathbb{S}^1$ is an example of path-connected space that is not contractible.

\begin{thm}{Exercise}
Let $\c{X}$ be a contractible space.

\begin{subthm}{}
Show that any two continuous maps from a topological space to $\c{X}$ are homotopic.
\end{subthm}

\begin{subthm}{}
Show that any two continuous maps from $\c{X}$ to a path-connected space are homotopic.
\end{subthm}

\end{thm}

\section{Homotopy type}

Two topological spaces $\c{X}$ and $\c{Y}$ have the \emph{same homotopy type} (briefly $\c{X}\sim\c{Y}$) if there are continuous maps
$f\: \c{X}\to \c{Y}$ and $h\:\c{Y}\to \c{X}$ such that $h\circ f\sim \id_\c{X}$ and $f\circ h\sim \id_\c{Y}$.

\begin{thm}{Exercise}
Show that a topological $\c{X}$ is contractible if and only if it has the same homotopy type with the one-point space.
\end{thm}

\section{Homotopy of paths}

Suppose that $f_t\:[0,1]\to\c{X}$ is homotopy of paths relative to its ends;
that is, $f_t(0)$ and $f_t(1)$ do not depend on $t$.
Then we say that $f_0$ is \emph{homotopic} to  $f_1$, briefly $f_0\sim f_1$.
This is shortcut notation --- formally speaking, we had to say \emph{homotopic rel. $\{0,1\}$} and write $f_0\sim f_1\rel{\{0,1\}}$,

Since $\sim$ is an equivalence relation (\ref{ex:hom-eq}), we can talk about the equivalence class of a path $f$ that will be called its \emph{homotopy class};
it will be denoted by $[f]$.

\section{Fundamental group}

A path $f\:[0,1]\to\c{X}$ is called \emph{loop} based at $p\in\c{X}$ if $f(0)=f(1)=p$.

Suppose $f$ is a loop based at $p\in \c{X}$.
Recall that $[f]$ denotes the homotopy class of $f$; 
note that $[f]$ is a subset of all loops in $\c{X}$.

Let us define multiplication of homotopy classes by 
\[[f]\cdot [g]=[f*g];\]
that is, \textit{the product of homotopy classes of $f$ and $g$ is the homotopy class of the product $f*g$.}

Denote by $\pi_1(\c{X},p)$ the set of all homotopy classes of loops at $p$.

\begin{thm}{Theorem}
 $\pi_1(\c{X},p)$ with the introduced multiplication is a group.
\end{thm}

It will be done by checking every condition in the definition of group.
Let us reformulate these conditions using the definition of homotopy class.
Recall that $\bar f$ denoted the path $f$ with revered time; see Section~\ref{sec:Operations on paths}.

\parit{Proof.}
Denote by $\eps_p$ the constant loop at $p$ in $\c{X}$;
that is, $\eps_p(t)=p$ for any $t$.
Note that conditions in the definition of group follow from the next three conditions for any loops $f$, $g$, and $h$ based at $p$ in $\c{X}$.
\begin{enumerate}[(i)]
\item $f*\eps_p\sim \eps_p*f\sim f$;
\item $f*\bar f\sim\bar f* f\sim e$;
\item $(f*g)*h\sim f*(g*h)$.
\end{enumerate}
Slightly more general versions of these statements provided by the claims \ref{clm:neutral}, \ref{clm:f-bar-f}, and \ref{clm:assoc}.
\qeds

\section{Three claims}

Each of the following claims proved by explicit construction of the needed homotopy.
Each time the homotopy constructed as a composition $h\circ s_\tau(t)$, where $h$ is a fixed path and  $s_\tau$ is a one-parameter family of functions $[0,1]\to [0,1]$.
The graphs of $s_\tau$ provide more intuitive descriptions of the families;
the formulas presented just to make it formally correct.

\begin{thm}{Claim}\label{clm:neutral}
Suppose $f$ is a path from $x$ to $y$, then
\[\eps_x*f\sim f*\eps_y\sim  f.\]
\end{thm}

\begin{wrapfigure}[4]{r}{30mm}
\centering
\vskip-5mm
\includegraphics{mppics/pic-30}
\end{wrapfigure}

\parit{Proof.}
Consider the function
\[s_\tau(t)=
\begin{cases}
2\cdot \tau \cdot t&\text{if}\ t\le \tfrac12,
\\
2\cdot\tau-1+2\cdot (1-\tau) \cdot t&\text{if}\ t\ge \tfrac12.
\end{cases}
\]
Observe that 
\begin{align*}
f(s_0(t))&=\eps_x*f(t),
\\
f(s_{\frac12}(t))&=f(t),
\\
f(s_1(t))&=f(t)*\eps_y
\end{align*}
for any $t$.
Since $h_\tau(t)=f(s_\tau(t))$ is a homotopy, the claim follows.
\qeds

\begin{thm}{Claim}\label{clm:f-bar-f}
Suppose $f$ is a path from $x$ to $y$, then
\[f*\bar f\sim \eps_x
\quad\text{and}\quad
\bar f*f\sim \eps_y.\]
\end{thm}

{

\begin{wrapfigure}[12]{r}{30mm}
\centering
\includegraphics{mppics/pic-31}
\end{wrapfigure}

\parit{Proof.}
Consider the function
\[s_\tau(t)=
\begin{cases}
2\cdot \tau \cdot t&\text{if}\ t\le \tfrac12,
\\
1-2\cdot \tau \cdot t&\text{if}\ t\ge \tfrac12.
\end{cases}
\]
Observe that $f(s_1(t))=f*\bar f(t)$ for any $t$.
Therefore $h_\tau(t)=f(s_\tau(t))$ is a homotopy from $\eps_x$ to $f*\bar f$.
It proves the first statement.



To prove the second statement one has to redefine $s_\tau$ as
\[s_\tau(t)=
\begin{cases}
1-2\cdot \tau \cdot t&\text{if}\ t\le \tfrac12,
\\
1-2\cdot\tau + 2\cdot \tau \cdot t&\text{if}\ t\ge \tfrac12,
\end{cases}
\]
and observe that $f(s_1(t))=\bar f*f(t)$ for any $t$.
\qeds

}

\begin{thm}{Claim}\label{clm:assoc}
Suppose $f$ is a path from $x$ to $y$, 
$g$ is a path from $y$ to $z$,
and
$h$ is a path from $z$ to $w$,
then
\[(f*g)*h\sim f*(g*h).\]
\end{thm}

\begin{wrapfigure}[3]{r}{30mm}
\centering
\vskip-4mm
\includegraphics{mppics/pic-32}
\end{wrapfigure}

\parit{Proof.}
Consider the function $s_\tau$ defined by 
\[s_\tau(t)=
\begin{cases}
\tfrac{1+\tau}2\cdot t&\text{if}\ t\le \tfrac12,
\\
\tfrac{\tau-1}4+t&\text{if}\ \tfrac 34\ge t\ge \tfrac12,
\\
\tau-1+ (2-\tau)\cdot t&\text{if}\  t\ge \tfrac34.
\end{cases}
\]
Note that $s_1(t)=t$,
and therefore \[f*(g*h)(t)\z=f*(g*h)(s_1(t))\] for any $t$.
Further, 
\[(f*g)*h(t)\z=f*(g*h)(s_0(t))\] for any $t$.
It remains to observe that $f*(g*h)(s_\tau(t))$ is the needed homotopy.
\qeds
