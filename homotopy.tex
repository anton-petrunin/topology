\chapter{Topology meets algebra}

In the previous chapter we gave a construction of the fundamental group for a given topological space with marked point.
This construction is one of bridges between topology and abstract algebra.
It makes possible to reformulate a topological problem on using the language of abstract algebra and sometimes the other way around.
In order to be useful the construction should predictably behave with respect to change of topological space.
In this chapter we discuss ...

\section{Induced homomorphism}

\begin{thm}{Claim}\label{clm:ind-hom}
Let $\phi\:\c{X}\to \c{Y}$ be a continuous map and $\phi(p)=q$.
Suppose that $f_0$ and $f_1$ are loops bases at $p$ in $\c{X}$.
Then

\begin{subthm}{clm:ind-hom-prod}
$\phi\circ(f_0*f_1)=(\phi\circ f_0)*(\phi\circ f_1)$,
\end{subthm}

\begin{subthm}{clm:ind-hom-homotopy}
if $f_0\sim f_1$, then $\phi\circ f_0\sim\phi\circ f_1$.
\end{subthm}

\end{thm}

\parit{Proof; \ref{SHORT.clm:ind-hom-prod}.}
Applying the definition of the product of paths and composition of maps to $\phi\circ(f_0*f_1)$ and $(\phi\circ f_0)*(\phi\circ f_1)$ we get exactly the same expression:
\begin{align*}
\begin{cases}
\phi\circ f_0(t)&\text{if}\ t\le \tfrac12,
\\
\phi\circ f_1(t)&\text{if}\ t\ge \tfrac12.
\end{cases}
\end{align*}
Hence \ref{SHORT.clm:ind-hom-prod} follows.

\parit{\ref{SHORT.clm:ind-hom-homotopy}.}
Observe that if $f_\tau$ is a homotopy from $f_0$ to $f_1$, then $\phi\circ f_\tau$ is a homotopy from $\phi\circ f_0$ to $\phi\circ f_1$.
Hence \ref{SHORT.clm:ind-hom-homotopy} follows.
\qeds

\begin{thm}{Corollary}
Let $\phi\:\c{X}\to \c{Y}$ be a continuous map between topological spaces;
suppose $\phi(p)=q$.
Given a loop $f$ with base at $p$ in $\c{X}$, the composition $\phi\circ f$ is a loop with base at $q$ in $\c{Y}$.
Moreover, 
$\phi_*\:[f]\mapsto [\phi\circ f]$
defines a homomorphism $\phi_*\:\pi_1(\c{X},p)\to \pi_1(\c{Y},q)$.
\end{thm}



\begin{thm}{Exercise}
Consider continuous maps 
$\c{X}\stackrel{\phi}{\to}\c{Y}\stackrel{\psi}{\to}\c{Z}$ between topological spaces.
Show that $\psi_*\circ\phi_*=(\psi\circ\phi)_*$.
\end{thm}

\section{Dependence on base point}

\begin{thm}{Theorem}
Let $p$ and $q$ be two points in a topological space $\c{X}$.
Suppose there is a path $h$ from $p$ to $q$, then 
the fundamental groups $\pi_1(\c{X},p)$ and $\pi_1(\c{X},q)$ are isomorphic.
\end{thm}

Recall that product of paths is not associative;
that is, we might have that $(f*g)*h\ne f*(g*h)$ for some paths $f,g,h$ such that both sides of the equation are defined.
In other words we have to fully parenthesize the products of paths.
If the product is not parenthesized we assume that the product is taken in the usual order;
that is,
\[f*g*h*k\df((f*g)*h)*k.\]

\parit{Proof.}
Suppose $f$ is a loop based at $p$.
Note that $\bar h*f*h$ is a loop at $q$.
Moreover, the map $f\mapsto \bar h*f*h$ induces a homomorphism $u_h\:\pi_1(M,p)\to\pi_1(M,q)$.

Indeed, suppose $f_\tau$ is a homotopy of loops at $p$.
Then $\bar h*f_\tau*h$ is a homotopy of loops at $q$.
It follows that the map 
\[u_h\:[f]\mapsto [\bar h*f*h]\]
is defined; that is, the right-hand side does not depend on the choice of loop $f$ in the homotopy class $[f]$. 

Further, if $f$ and $g$ are loops based at $p$, then \ref{clm:neutral}, \ref{clm:f-bar-f}, and  \ref{clm:assoc} imply that
\begin{align*}
(\bar h*f*h)*(\bar h*g*h)&\sim \bar h*f*(h*\bar h)*g*h\sim
\\
&\sim \bar h*(f*\eps_p)*g*h\sim 
\\
&\sim \bar h*(f*g)*h.
\end{align*}
Whence the map $u_h\:\pi_1(M,p)\to \pi_1(M,q)$ is a homomorphism;
that is, 
\[u_h([f]\cdot [g])=u_h[f]\cdot u_h[g]\quad\text{for any}\quad [f],[g]\in \pi_1(\c{X},p).\]

The same argument shows that $u_{\bar h}\: \pi_1(M,q)\to \pi_1(M,p)$ defined by 
\[u_{\bar h}\:[k]\mapsto [h*(k*\bar h)]\]
is a homomorphism.

Finally note that 
\begin{align*}
h*(\bar h*f*h)*\bar h&\sim (h*\bar h)*f*(h*\bar h)\sim
\\
&\sim \eps_p*f*\eps_p\sim
\\
&\sim f
\end{align*}
for any loop $f$ based at $p$.
Therefore, $u_{\bar h}$ is a left inverse of $u_h$.
The same way we show that $u_{h}$ is a left inverse of $u_{\bar h}$.
It follows that $u_h$ is an isomorphism.
\qeds

According to the theorem, the fundamental group (more precisely its \textit{isomorphism class}) of path-connected space does not depend on its base point.
Therefore, for a path-connected space $\c{X}$ we do not need to specify the base point of its fundamental group; so we could write $\pi_1(\c{X})$ instead of $\pi_1(\c{X},p)$ (if we think about the group up to isomorphism).

\begin{thm}{Exercise}\label{ex:uphi}
Let $\phi_\tau\:\c{X}\to\c{Y}$ be a homotopy.
Suppose that $q_0=\phi_0(p)$ and $q_1=\phi_1(p)$; 
consider the path from $q_0$ to $q_1$ defined by $h(t)=\phi_\tau(p)$.
Show that 
\[u_h\circ \phi_{0*}=\phi_{1*}.\]
\end{thm}

\begin{thm}{Exercise}
Suppose that path-connected topological spaces $\c{X}$ and $\c{Y}$ have the same homotopy type.
Use \ref{ex:uphi} to show that their fundamental groups are isomorphic.
\end{thm}



\begin{thm}{Exercise}
Let $\c{X}$ and $\c{Y}$ be two path-connected topological spaces.
Choose points $p\in \c{X}$ and $q\in \c{Y}$.
Consider the projections $\phi\:\c{X}\times\c{Y}\to \c{X}$ and $\psi\:\c{X}\times\c{Y}\to \c{Y}$
and their induced homomorphisms
$\phi_*\:\pi_1(\c{X}\times\c{Y},(p,q))\to \pi_1(\c{X},p)$
and $\psi_*\:\pi_1(\c{X}\times\c{Y},(p,q))\to \pi_1(\c{Y},q)$.
Define $\Phi\:\pi_1(\c{X}\times\c{Y},(p,q))\to \pi_1(\c{X},p)\times \pi_1(\c{Y},q)$ by
\[\Phi\:\alpha\mapsto (\phi_*(\alpha),\psi_*(\alpha))\]
for any $\alpha\in \pi_1(\c{X}\times\c{Y},(p,q))$.
Note that the map $\Phi$ is a homomorphism.

\begin{subthm}{}
Show that $\Phi$ is a monomorphism;
that is, if $\Phi(\alpha)=\Phi(\beta)$ for some $\alpha,\beta\in \pi_1(\c{X}\times\c{Y},(p,q))$, then 
$\alpha=\beta$.
\end{subthm}

\begin{subthm}{}
Show that $\Phi$ is an epimorphism;
that is, for any $\gamma\in \pi_1(\c{X},p)\z\times\pi_1(\c{Y},q)$ there is $\alpha\in\pi_1(\c{X}\times\c{Y},(p,q))$ such that $\Phi(\alpha)=\gamma$.
\end{subthm}

Conclude that $\pi_1(\c{X}\times\c{Y})$ is isomorphic to $\pi_1(\c{X})\times\pi_1(\c{Y})$.
\end{thm}

\begin{thm}{Advanced exercise}
Let $\c{X}$ be a topological space.
Consider the quotient space $\c{Y}=(\c{X}\times \c{X})/\sim$ where $\sim$ is the minimal equivalence relation such that $(x,y)\sim (y,x)$.
Show that the fundamental group $\pi_1(\c{Y},(x,x))$ is commutative for any $x\in \c{X}$.
\end{thm}


\section{Definitions}

Assume $f_t\:\c{X}\to \c{Y}$ be a one-parameter family of maps, $t\in [0,1]$.
If the map $[0,1]\times \c{X}\to \c{Y}$, defined as $(t,x)\mapsto f_t(x)$ is continuous, then $f_t$ is called a \index{homotopy}\emph{homotopy} of maps from $\c{X}$ to $\c{Y}$.

If $f_t(a)$ is independent of $t$ for any point $a$ in some subset $A\subset \c{X}$, then we say that  $f_t$ is a {}\emph{homotopy relative to $A$}. 

Two maps $g,h\:\c{X}\to \c{Y}$ are called {}\emph{homotopic} (briefly $g\sim h$)
if there is a homotopy $f_t\:\c{X}\to \c{Y}$ such that $g=f_0$ and $h=f_1$.

Two maps $g,h\:\c{X}\to \c{Y}$ are called \emph{homotopic relative to} $A$ (briefly, $g\sim h\rel A$)
if there is a homotopy $f_t\:\c{X}\to \c{Y}$ relative to $A$ such that $g=f_0$ and $h=f_1$.

\begin{thm}{Exercise}\label{ex:hom-eq}
Show that ``$\sim$''  defines equivalence relations on the set of continuous maps between given topological spaces.
That is, for any continuous maps $f,g,h\:\c{X}\to \c{Y}$, we have 
\begin{itemize}
\item $f\sim f$, 
\item if $f\sim h$ then $h\sim f$, and
\item if $f\sim g$ and $g\sim h$ then $f\sim h$.
\end{itemize}
Do the same for ``$\sim\rel A$''.
\end{thm}

Suppose that $f_\tau\:[0,1]\to\c{X}$ is homotopy of paths relative to its ends;
that is, $f_\tau(0)$ and $f_\tau(1)$ do not depend on $t$.
Then we say that $f_0$ is \index{homotopy of paths}\emph{homotopic} to $f_1$, briefly $f_0\sim f_1$.
This is shortcut notation --- formally speaking, we had to say \textit{homotopic rel. $\{0,1\}$} and write $f_0\sim f_1\rel{\{0,1\}}$,

