\chapter{Finite spaces}

Finite spaces provide a source of examples of topological spaces.
They also have some distinguishing properties, which we will discuss in this chapter.

\section{Arbitrary unions and intersections}

Let $\c{F}$ be a finite topological space;
that is, a finite set equipped with a topology.

Note that $\c{F}$ contains finitely many distinct open sets.
Since intersection of finite collection of open sets is open,
we can conclude that arbitrary intersection of open sets in $\c{F}$ is open.

Similarly, arbitrary union of closed sets in $\c{F}$ is closed.

In other words, for the finite topological spaces $\c{F}$, open and closed sets have the same properties.
In particular one could define a new topology on $\c{F}$ by declaring all open sets to be closed and the other way around.

These observations are the key properties which distinguish finite spaces from the rest of topological spaces.

\section{Preorder}

Let $\c{F}$ be a finite topological space.

Given a point $x\in \c{F}$,
denote by $\bar x$ the closure of one-point set $\{x\}$.

For two points $x,y\in \c{F}$ write $x\succcurlyeq y$ if $\bar x\supset \bar y$.
The introduced relation ``$\succcurlyeq$'' behaves very much as ``$\ge$'' --- formally speaking it defines a \index{preorder}\emph{preorder} on $\c{F}$;
that is, for any $x,y,z\in\c{F}$ the following conditions are fulfilled: 
\begin{itemize}
\item $x\succcurlyeq x$ (reflexivity).
\item if $x\succcurlyeq y$ and $y\succcurlyeq z$, then $x\succcurlyeq z$ (transitivity).
\end{itemize}

Conversely, if $\c{F}$ is a finite set with preorder ``$\succcurlyeq$'',
one can define topology on $\c{F}$ by declaring that for any $\bar x=\set{y\in \c{F}}{x\succcurlyeq y}$ and all its finite intersections and unions are closed.
Thus the partial order is an equivalent way to describe topology.




