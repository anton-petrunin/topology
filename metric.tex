\chapter{Metric spaces}

In this chapter we discuss \emph{metric spaces} --- 
a motivating example that will guide us toward the definition of \emph{topological spaces} --- the main object of topology.

Examples of metric spaces were considered for thousands of years,
but the first general definition was given only in 1906 by Maurice Fr\'echet.

\section{Definition}

In the following definition we grab together the most important properties of the intuitive notion of \emph{distance}.

\begin{thm}{Definition}\label{def:metric-space}
Let $\c X$ be a nonempty set 
with a function 
that returns a real number, denoted as $|x-y|$,
for any pair $x,y\in\c X$.
Assume that the following conditions are satisfied for any
$x,y,z\in \c X$:

\begin{subthm}{def:metric-space:a}
Positiveness: 
$$|x-y|\ge 0.$$
\end{subthm}

\begin{subthm}{def:metric-space:b}
Identity of indiscernibles:  
$$x=y\quad\text{if and only if}\quad |x-y|=0.$$
\end{subthm}

\begin{subthm}{def:metric-space:c}
Symmetry: $$|x-y| = |y- x|.$$
\end{subthm}

\begin{subthm}{def:metric-space:d}
Triangle inequality: 
$$  |x- y| + |y- z| \ge |x-z|.$$
\end{subthm}

In this case, we say that $\c X$ is a \index{metric space}\emph{metric space} 
and the function 
\[(x,y)\mapsto |x-y|\] 
is called a \emph{metric}. 

The elements of $\c X$ are called \index{point}\emph{points} of the metric space.
Given two points $x,y\z\in \c X$, 
the value $|x-y|$ is called \index{distance}\emph{distance} from $x$ to~$y$.
\end{thm}

Note that for two points in a metric space  the difference between points $x-y$ may have no meaning,
but $|x-y|$ always has the meaning defined above.

Typically, we consider only one metric on set, 
but if few metrics are needed, we can distinguish them by an index, say  $|x-y|_{\bullet}$ or $|x-y|_{239}$.
If we need to emphasize that the distance is taken in the metric space $\c{X}$ we write $|x-y|_{\c X}$ instead of $|x-y|$.

\section{Examples}\label{sec:examples(metric-spaces)}

Let us give a few examples of metric spaces.

\begin{itemize}
\item\emph{Discrete space.}\label{page:Discrete metric} Let $\c X$ be an arbitrary set. 
For any $x,y\z\in\c X$, 
set $|x-y|\z=0$ if $x=y$ and $|x-y|=1$ otherwise.
This metric is called \index{discrete metric}\emph{discrete metric} on $\c X$ and the obtained metric space is called \emph{discrete}.
\item\index{real line}\emph{Real line.} 
Set of all real numbers ($\mathbb{R}$) with metric defined as 
$|x-y|$. 
(Unless it is stated othewise, the real line $\mathbb{R}$ will be considered with this metric.) 
\item {}\emph{Metrics on the plane.}\label{page:Metrics on the plane}
Let us denote by $\mathbb{R}^2$ the set of all pairs $(x,y)$ of real numbers.
Consider two points $p=(x_p,y_p)$ and $q=(x_q,y_q)$  in~$\mathbb{R}^2$.
One can equip $\mathbb{R}^2$ with the following metrics:
\begin{itemize}
\item\index{Euclidean metric}\emph{Euclidean metric,} denoted by $|p-q|_2$  and defined as
$$|p-q|_2=\sqrt{(x_p-x_q)^2+(y_p-y_q)^2}.$$
\item\label{Manhattan plane}\index{Manhattan plane}\emph{Manhattan metric,}  
$$|p-q|_1=|x_p-x_q|+|y_p-y_q|.$$
\item{}\emph{Maximum metric,} 
$$|p-q|_\infty=\max\{|x_p-x_q|,|y_p-y_q|\}.$$
\end{itemize}
\end{itemize}

\begin{thm}{Exercise}\label{ex:d1+d2+dinfty}
Prove that (a)~$|{*}-{*}|_2$; (b)~$|{*}-{*}|_1$ and (c)~$|{*}-{*}|_\infty$ are metrics on $\mathbb{R}^2$.
\end{thm}

\begin{thm}{Exercise}\label{ex:not-a-metric}
Show that 
\[|x-y|_{\natural}=(x-y)^2\]
is not a metric on $\RR$.
\end{thm}

\begin{thm}{Exercise}\label{ex:metric}
Show that if $(x,y)\mapsto |{x}-{y}|$ is a metric, then so is 
\[(x,y)\mapsto |{x}-{y}|_{\max}=\max\{1,|{x}-{y}|\}.\] 
\end{thm}

\section{Subspaces}\label{sec:subspaces-metric}

Any subset $\c{A}$ of metric space $\c{X}$ forms a metric space on its own;
it is called \emph{subspace} of $\c{X}$.
This construction produces many more examples of metric spaces. 
For example, the disc
\begin{align*}
\DD^2&=\set{(x,y)\in\RR^2}{x^2+y^2<1}
\intertext{and the circle}
\SS^1&=\set{(x,y)\in\RR^2}{x^2+y^2=1},
\end{align*}
are metric spaces with metrics taken from the Euclidean plane.
Similarly, the interval $[0,1)$ is a metric space with metric taken from $\RR$.


\section{Continuous maps}

Recall that a real-to-real function $f$ is called \emph{continuous} 
if for any $x\in \RR$ and any $\eps > 0$ there exists $\delta > 0$ such that $|f(x)-f(y)|<\eps$, whenever $|x-y|<\delta$. 

This definition can be used for the functions defined on Euclidean space if $|x-y|$ denotes the Euclidean distance $|x-y|_2$ between the points $x$ and $y$.
It admits the following straightforward generalization to \emph{metric spaces}:

\begin{thm}{Definition}\label{def:cont-epsilon-delta}
A function $f\:\c{X}\to\c{Y}$ between metric spaces is called \emph{continuous} 
if for any  $x\in \c{X}$ and any $\eps>0$ there exists $\delta>0$ such that 
$|f(x)-f(y)|_{\c{Y}}<\eps$,
for any
$y\in \c{X}$
such that
$|x-y|_{\c{X}}<\delta$.
\end{thm}

\begin{thm}{Exercise}\label{ex:dist-cont}
Let $\c{X}$ be a metric space and $z\in \c{X}$ be a fixed point.
Show that the function 
$$f(x)\df|x-z|$$ 
is continuous.
\end{thm}

\begin{thm}{Exercise}\label{ex:comp+cont}
Let $\c{X}$, $\c{Y}$ and $\c{Z}$ be metric spaces.
Assume that the functions $f\:\c{X}\to\c{Y}$
and $g\:\c{Y}\to\c{Z}$ are continuous,
and 
\[h=g\circ f\:\c{X}\to \c{Z}\] is its composition;
that is, $h(x)=g(f(x))$ for any $x\in \c{X}$.
Show that $h\:\c{X}\to\c{Z}$ is continuous at any point.
\end{thm}

\begin{thm}{Exercise}\label{ex:isom-cont}
Show that any distance-preserving map is continuous.

More precisely, if $f\:\c{X}\to\c{Y}$ is a map between metric space such that
\[|x-x'|_{\c X}=
|f(x)-f(x')|_{\c Y}\]
for any $x,x'\in \c{X}$, then 
$f$ is continuous.
\end{thm}

\begin{thm}{Exercise}\label{ex:to-descrete}
Let $\c{X}$ be a discrete metric space (defined in \ref{sec:examples(metric-spaces)})
and $\c{Y}$ be arbitrary metric space.
Show that for any function $f\:\c{X}\to\c{Y}$ is continuous.
\end{thm}



\begin{thm}{Advanced exercise}
Construct a continuous function 
\[f\:[0,1]\z\to [0,1]\] that takes every value in $[0,1]$ 
an infinite number of times.
\end{thm}


\section{Balls}

Let $\c{X}$ be a metric space, 
$x$ is a point in $\c{X}$ 
and $r$ is a positive real number.
The set of points in $\c{X}$ which lies on the distance smaller than $r$ is called \emph{open ball} of radius $r$ centered at $x$.
It is denoted as $\Ball(x,r)$ 
or $\Ball(x,r)_{\c{X}}$ if we need to emphasize that it is taken in the space $\c{X}$.

The ball $\Ball(x,r)$ is also called $r$-neighborhood of $x$.

Analogously we may define \emph{closed balls} 
\[\cBall[x,r]=\cBall[x,r]_{\c{X}}=\set{y\in \c{X}}{|x-y|\le r}.\] 

\begin{thm}{Exercise}\label{ex:d1+d2+dinfty-balls}
Sketch the unit balls for the metrics $|{*}-{*}|_1$, $|{*}-{*}|_2$ and $|{*}-{*}|_\infty$ defined right before Exercise~\ref{ex:d1+d2+dinfty}.
\end{thm}

\begin{thm}{Exercise}
Assume $\Ball(x,r)$  and $\Ball(y,R)$ is a pair of balls in a metric space 
and $\Ball(x,r)\subsetneq \Ball(y,R)$.
Show that $r<2\cdot R$.

Give an example of a metric space and a pair of balls as above such that $r> R$.
\end{thm}

Let us reformulate the definition of continuous map (\ref{def:cont-epsilon-delta}) using the introduced notion of ball.

\begin{thm}{Definition}\label{def:cont-balls}
A function $f\:\c{X}\to\c{Y}$ between metric spaces is called \emph{continuous} if for any $x\in \c{X}$ and any $\eps>0$ there exists $\delta>0$ such that 
\[f(\Ball(x,\delta)_{\c{X}})\subset\Ball(f(x),\eps)_{\c{Y}}.\]

\end{thm}

\begin{thm}{Exercise}
Prove the equivalence of definitions \ref{def:cont-epsilon-delta} and \ref{def:cont-balls}.
\end{thm}



\section{Open sets}

\begin{thm}{Definition}\label{def:open}
A subset $V$ in a metric space $\c{X}$ is called \emph{open} if for any $x\in V$ there is $\eps>0$ such that $\Ball(x,\eps)\subset V$.
\end{thm}

In other words, $V$ is open if, together with each point, $V$ contains its $\eps$-neighborhood for some $\eps>0$.
For example, any set in a discrete metric space is open since together with any point it contains its $1$-neighborhood.
Further the set of positive real numbers
\[(0,\infty)=\set{x\in\RR}{x>0}\] 
is open; indeed, together with each point $x>0$ it contains its $x$-neighborhood.
On the other hand, the set of nonnegative reals 
\[[0,\infty)=\set{x\in\RR}{x\ge0}\]
is not open since there are negative numbers in any neighborhood of~$0$.

\begin{thm}{Exercise}\label{ex:ball-is-open}
Show that any ball in a metric space is open.
\end{thm}

\begin{thm}{Exercise}\label{ex:union-of-balls}
Show that a set in a metric space is open if and only if it is a union of balls. 
\end{thm}

\begin{thm}{Exercise}\label{ex:open-union} Show that the union of an arbitrary collection of open sets is open.
\end{thm}

\begin{thm}{Exercise}\label{ex:open-intersection}
Show that the intersection of two open sets is open.
\end{thm}

\begin{thm}{Exercise}
Give an example of metric space $\c{X}$ and an infinite sequence of open sets $V_1,V_2,\dots$
such that the intersection
\[\bigcap_nV_n\]
is not open.
\end{thm}

\begin{thm}{Exercise}\label{ex:d1+d2+dinfty-open}
Show that the metrics $|{*}-{*}|_1$, $|{*}-{*}|_2$ and $|{*}-{*}|_\infty$ (defined in \ref{sec:examples(metric-spaces)})
give rise to the same open sets in $\RR^2$.
That is, if $V\subset \RR^2$ is open for one of these metrics, then it is open for the others.
\end{thm}

\section{Gateway to topology}

The following result is the main gateway to topology.
It says that continuous maps can be defined entirely in terms of open sets.

\begin{thm}{Proposition}\label{prop:cont-open}
A function $f\:\c{X}\to\c{Y}$ 
between two metric spaces is continuous 
if and only if for any open set $W\subset\c{Y}$ 
its inverse images
\[f^{-1}(W)=\set{x\in \c{X}}{f(x)\in W}\]
is open.
\end{thm}

Note that proposition says nothing about the images of open sets. In fact, before going into proof it would be useful to solve the following exercise.

\begin{thm}{Exercise}\label{ex:image-of-open}
Give an example of a continuous $f\:\RR\to\RR$ and an open set $V\subset \RR$ such that the image $f(V)\subset \RR$ is not open.
\end{thm}

The formulation of the proposition contains ``if and only if''
and the proof breaks into two parts ``if''-part and ``only if''-part.

\parit{Proof; ``only-if'' part.} 
Let $W\subset\c{Y}$ be an open set and $V=f^{-1}(W)$.
Choose $x\in V$; note that so $f(x)\in W$.

Since $W$ is open, 
\[\Ball(f(x),\eps)_{\c{Y}}\subset W
\eqlbl{eq:BinU}\] 
for some $\eps>0$.

Since $f$ is continuous, by Definition~\ref{def:cont-balls}, there is $\delta>0$ such that
\[f(B(x,\delta)_{\c{X}})\subset \Ball(f(x),\eps)_{\c{Y}}.\]
It follows that together with any point $x\in V$, the set $V$ contains $B(x,\delta)$;
that is, $V$ is open.

\parit{``If'' part.} Fix $x\in \c{X}$ and $\eps>0$.
According to Exercise~\ref{ex:ball-is-open}, 
\[W=\Ball(f(x),\eps)_{\c{Y}}\] is an open set in $\c{Y}$.
Therefore its inverse image $f^{-1}(W)$ is open.

Clearly $x\in f^{-1}(W)$.
By the definition of open set (\ref{def:open})
\[\Ball(x,\delta)_{\c{X}}\subset f^{-1}(W)\] for some $\delta>0$.
Or equivalently
\[f(\Ball(x,\delta)_{\c{X}})\subset W=\Ball(f(x),\eps)_{\c{Y}}.\]
Hence the ``if''-part follows.\qeds

\section{Limits}

\begin{thm}{Definition}\label{def:limit-metric}
Let $x_1,x_2,\dots$ be a sequence of points in a metric space $\c{X}$.
We say the sequence $x_n$ converges to a point $x_\infty\in \c{X}$ if 
\[|x_\infty-x_n|_{\c{X}}\to 0\quad\text{as}\quad n\to \infty.\]
In this case, we say that the sequence $(x_n)$ is a converging sequence and $x_\infty$ is its limit; the latter will be written as \[x_\infty=\lim_{n\to\infty}x_n\]
\end{thm}

Note that we defined the convergence of points in a metric space using the convergence of real numbers $d_n=|x_\infty-x_n|_{\c{X}}$, which we assume to be known.

\begin{thm}{Exercise}
Show that any sequence of points in a metric space has at most one limit. 
\end{thm}


\begin{thm}{Exercise}\label{ex:continuous-limit}
Let $f\:\c{X}\to\c{Y}$ be a function between metric spaces.
Show that $f$ is continuous if and only if the following condition holds:
\begin{itemize}
 \item If $x_n\to x_\infty$ as $n\to\infty$ in $\c{X}$, then the sequence $y_n=f(x_n)$ converges to $y_\infty=f(y_\infty)$ as $n\to\infty$ in $\c{Y}$.
\end{itemize}

\end{thm}


\section{Closed sets}

Let $A$ be a set in a metric space $\c{X}$.
A point $x\in \c{X}$ is a \emph{limit point} of $A$ if there is a sequence $x_n\in A$ such that $x_n\to x$ as $n\to\infty$.%
\footnote{Sometimes limit points are defined, assuming in addition that $x_n\z\ne x$ for any~$n$ --- we do \textit{not} follow this convention.}

The set of all limit points of $A$ is called the \emph{closure} of $A$ and denoted as $\bar A$.
Note that $\bar A\supset A$;
indeed, any point $x\in A$ is a limit point of the constant sequence $x_n=x$.

If $\bar A= A$, then the set is called \emph{closed}.

\begin{thm}{Exercise}
Give an example of a subset $A\subset\RR$ that is neither closed nor open. 
\end{thm}


\begin{thm}{Exercise}
Show that closure of any set in metric space is a closed set;
that is, $\bar{\bar A}= \bar A$.
\end{thm}

\begin{thm}{Exercise}
Show that a subset $Q$ in a metric space $\c{X}$ is closed if and only if its complement $\c{X}\backslash Q$ is open.
\end{thm}

The following exercise is a partial case of the \emph{Tietze--Urysohn extension theorem}.

\begin{thm}{Advanced exercise}\label{ex:tietze}
Let $Q$ be a closed subset in a matric space $\c{X}$.
Show that any continuous function $Q\to\RR$ can be extended to a continuous function $\c{X}\to\RR$.
\end{thm}

